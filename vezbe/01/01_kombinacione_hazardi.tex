%========================
% A4 Article Template
%========================
\documentclass[a4paper,11pt]{article}

% Load xcolor ONCE with all options you need
\usepackage[dvipsnames,table]{xcolor}

%---- Page layout & typography
\usepackage[a4paper,margin=2.5cm,headheight=15pt]{geometry}
\usepackage{microtype} % nicer text
\usepackage{setspace}  % optional line spacing
%\onehalfspacing        % uncomment for 1.5 spacing

% Add/replace these in your preamble
\usepackage[utf8]{inputenc}
\usepackage[T1]{fontenc}
\usepackage[serbian]{babel} % latin script variant
\usepackage{lmodern}

% If using XeLaTeX/LuaLaTeX, comment the three lines above and use:
% \usepackage{fontspec}
% \setmainfont{Latin Modern Roman}

%---- Flush floats
\usepackage{placeins} % preamble

%---- Karnaugh map
\usepackage[label=corner]{karnaugh-map}

%---- Math
\usepackage{amsmath,amssymb,mathtools}
\numberwithin{equation}{section} % eq numbers like (1.1)

%---- Colors & links
\definecolor{Primary}{HTML}{1F77B4} % custom palette
\definecolor{Accent}{HTML}{2CA02C}
\usepackage[colorlinks=true,linkcolor=Primary,citecolor=Primary,urlcolor=Primary]{hyperref}
\usepackage{cleveref} % smart refs: \cref{fig:...}

%---- Figures
\usepackage{graphicx}
\graphicspath{{Images/}} % put images in ./Images
\usepackage{subcaption} % subfigures
\usepackage{caption}
\captionsetup{font=small,labelfont=bf}

%---- Tables
\usepackage{booktabs}   % professional tables
\usepackage{tabularx}   % tables with flexible width
\usepackage{arydshln}   % adds : for dashed vertical lines and \hdashline
\usepackage{multirow}   % optional, merged rows
\usepackage{siunitx}    % alignment for numbers/units
\sisetup{detect-all}

% for the table and logic symbols
\usepackage{array,makecell}
\usepackage{tikz}
\usetikzlibrary{circuits.logic.US,positioning}


%---- Lists (optional nicer lists)
\usepackage{enumitem}
\setlist{nosep}

%---- Header/Footer (optional)
\usepackage{fancyhdr}
\pagestyle{fancy}
\fancyhf{}
\lhead{\textit{Vežbe 1}}
\rhead{\thepage}

%---- Table settings
\renewcommand{\arraystretch}{1.25}
\renewcommand{\tabularxcolumn}[1]{m{#1}} % X behaves like m{...}
\newcolumntype{C}{>{\centering\arraybackslash}X} % centered X
\newcolumntype{M}[1]{>{\centering\arraybackslash}m{#1}}
%\setlength{\arrayrulewidth}{0.5pt} % optional thicker borders

\renewcommand{\contentsname}{Sadržaj}

%========================
% Document
%========================
\begin{document}


%====================================
% Custom title page (Serbian, A4)
%====================================
\begin{titlepage}
  \thispagestyle{empty}
  \begin{center}

    %--- Top (faculty/department)
    {\Large\bfseries
    UNIVERZITET U BEOGRADU -- ELEKTROTEHNIČKI FAKULTET\\
    KATEDRA ZA ELEKTRONIKU\par}

    \vspace{25mm}

    %--- Middle (course + subtitle)
    {\LARGE\bfseries DIGITALNA ELEKTRONIKA 1\par}
    \vspace{2mm}
    {\large Materijali za računske vežbe\par}

    \vspace{18mm}

    %--- Title block (exercise title + number)
    {\Large\bfseries
    SINTEZA KOMBINACIONIH MREŽA POMOĆU KOLA NISKOG 
    STEPENA INTEGRACIJE I STATIČKI HAZARDI\par}
    \vspace{3mm}
    {\large Vežbe 1\par}

    \vfill

%--- Bottom (authors table on left, date on right)
\begin{minipage}{0.62\linewidth}
  \raggedright
  \textbf{Autori:}\\[4pt]
  \small
  \begin{tabular}{@{}l l@{}}
    Nikola Petrović & \href{mailto:p.z.nikola@etf.bg.ac.rs}{\texttt{p.z.nikola@etf.bg.ac.rs}} \\
    Haris Turkmanović & \href{mailto:haris@etf.bg.ac.rs}{\texttt{haris@etf.bg.ac.rs}} \\
  \end{tabular}
\end{minipage}
\hfill
\begin{minipage}{0.34\linewidth}
  \raggedleft
  Beograd, \today
\end{minipage}

  \end{center}
\end{titlepage}

%========================

\tableofcontents
\newpage

%========================
\section{Uvod}
U okviru prvog dela ovih računskih vežbi obrađivaćemo praktične primere vezane za
sintezu kombinacionih mreža korišćenjem kola niskog stepena integracije. Pod kolima niskog
stepena integracija, za potrebe ovog kursa, podrazumevaćemo grupu osnovnih logičkih kola
predstavljenih u tabeli~\ref{tab:osnovna-kola}.



\begin{table}[!ht]
  \centering
  \caption{Pregled osnovnih logičkih kola}
  \label{tab:osnovna-kola}

  \begin{tabularx}{\linewidth}{| C | M{3.3cm} | C | M{3cm} |}
    \hline
    \rowcolor{gray!40}
    \textbf{Naziv} & \textbf{Simbol} & \textbf{Funkcija} & \textbf{Tabela istinitosti} \\
    \hline

    Invertor (NOT) &
    \begin{tikzpicture}[circuit logic US, x=1em, y=1em, baseline={(current bounding box.center)}]
      \node[not gate US, draw, minimum width=2.2em, logic gate inputs=nn] (g) {};
      \draw (g.input) -- ++(-0.9,0) node[left] {$A$};
      \draw (g.output) -- ++(0.9,0) node[right] {$Y$};
    \end{tikzpicture}
    & $Y=\lnot A = \bar{A}$ &
    {\scriptsize
      \begin{minipage}{\linewidth}\centering
        \vspace{3pt}
        \begingroup
        \rowcolors{0}{}{} % disable any active row coloring
        \begin{tabular}{|c|c|}
          \hline \rowcolor{gray!40} $A$ & $Y$ \\ \hline
          0 & 1 \\ \hline
          1 & 0 \\ \hline
        \end{tabular}
        \endgroup
        \vspace{3pt}
      \end{minipage}
    }\\ \hline

    Bafer (BUFFER) &
    \begin{tikzpicture}[circuit logic US, x=1em, y=1em, baseline={(current bounding box.center)}]
      \node[buffer gate US, draw, minimum width=2.2em, logic gate inputs=nn] (g) {};
      \draw (g.input) -- ++(-0.9,0) node[left] {$A$};
      \draw (g.output) -- ++(0.9,0) node[right] {$Y$};
    \end{tikzpicture}
    & $Y=A$ &
    {\scriptsize
      \begin{minipage}{\linewidth}\centering
        \vspace{3pt}
        \begingroup
        \rowcolors{0}{}{} % disable any active row coloring
        \begin{tabular}{|c|c|}
          \hline \rowcolor{gray!40} $A$ & $Y$ \\ \hline
          0 & 0 \\ \hline
          1 & 1 \\ \hline
        \end{tabular}
        \endgroup
        \vspace{3pt}
      \end{minipage}
    }\\ \hline

    I (konjunkcija, AND) &
    \begin{tikzpicture}[circuit logic US, x=1em, y=1em, baseline={(current bounding box.center)}]
      \node[and gate US, draw, minimum width=3em, logic gate inputs=nn] (g) {};
      \draw (g.input 1) -- ++(-0.9,0) node[left] {$A$};
      \draw (g.input 2) -- ++(-0.9,0) node[left] {$B$};
      \draw (g.output) -- ++(0.9,0) node[right] {$Y$};
    \end{tikzpicture}
    & $Y=A \land B = A \cdot B$ &
    {\scriptsize
      \begin{minipage}{\linewidth}\centering
        \vspace{3pt}
        \begingroup
        \rowcolors{0}{}{} % disable any active row coloring
        \begin{tabular}{|c|c|c|}
          \hline \rowcolor{gray!40} $A$ & $B$ & $Y$ \\ \hline
          0 & 0 & 0 \\ \hline
          0 & 1 & 0 \\ \hline
          1 & 0 & 0 \\ \hline
          1 & 1 & 1 \\ \hline
        \end{tabular}
        \endgroup
        \vspace{3pt}
      \end{minipage}
    }\\ \hline

    ILI (disjunkcija, OR) &
    \begin{tikzpicture}[circuit logic US, x=1em, y=1em, baseline={(current bounding box.center)}]
      \node[or gate US, draw, minimum width=3em, logic gate inputs=nn] (g) {};
      \draw (g.input 1) -- ++(-0.9,0) node[left] {$A$};
      \draw (g.input 2) -- ++(-0.9,0) node[left] {$B$};
      \draw (g.output) -- ++(0.9,0) node[right] {$Y$};
    \end{tikzpicture}
    & $Y=A \lor B = A + B$ &
    {\scriptsize
      \begin{minipage}{\linewidth}\centering
        \vspace{3pt}
        \begingroup
        \rowcolors{0}{}{} % disable any active row coloring
        \begin{tabular}{|c|c|c|}
          \hline \rowcolor{gray!40} $A$ & $B$ & $Y$ \\ \hline
          0 & 0 & 0 \\ \hline
          0 & 1 & 1 \\ \hline
          1 & 0 & 1 \\ \hline
          1 & 1 & 1 \\ \hline
        \end{tabular}
        \endgroup
        \vspace{3pt}
      \end{minipage}
    }\\ \hline

    NI (NAND) &
    \begin{tikzpicture}[circuit logic US, x=1em, y=1em, baseline={(current bounding box.center)}]
      \node[nand gate US, draw, minimum width=3em, logic gate inputs=nn] (g) {};
      \draw (g.input 1) -- ++(-0.9,0) node[left] {$A$};
      \draw (g.input 2) -- ++(-0.9,0) node[left] {$B$};
      \draw (g.output) -- ++(0.9,0) node[right] {$Y$};
    \end{tikzpicture}
    & $Y=\lnot(A \land B) = \overline{A \cdot B}$ &
    {\scriptsize
      \begin{minipage}{\linewidth}\centering
        \vspace{3pt}
        \begingroup
        \rowcolors{0}{}{} % disable any active row coloring
        \begin{tabular}{|c|c|c|}
          \hline \rowcolor{gray!40} $A$ & $B$ & $Y$ \\ \hline
          0 & 0 & 1 \\ \hline
          0 & 1 & 1 \\ \hline
          1 & 0 & 1 \\ \hline
          1 & 1 & 0 \\ \hline
        \end{tabular}
        \endgroup
        \vspace{3pt}
      \end{minipage}
    }\\ \hline

    NILI (NOR) &
    \begin{tikzpicture}[circuit logic US, x=1em, y=1em, baseline={(current bounding box.center)}]
      \node[nor gate US, draw, minimum width=3em, logic gate inputs=nn] (g) {};
      \draw (g.input 1) -- ++(-0.9,0) node[left] {$A$};
      \draw (g.input 2) -- ++(-0.9,0) node[left] {$B$};
      \draw (g.output) -- ++(0.9,0) node[right] {$Y$};
    \end{tikzpicture}
    & $Y=\lnot(A \lor B) = \overline{A + B}$ &
    {\scriptsize
      \begin{minipage}{\linewidth}\centering
        \vspace{3pt}
        \begingroup
        \rowcolors{0}{}{} % disable any active row coloring
        \begin{tabular}{|c|c|c|}
          \hline \rowcolor{gray!40} $A$ & $B$ & $Y$ \\ \hline
          0 & 0 & 1 \\ \hline
          0 & 1 & 0 \\ \hline
          1 & 0 & 0 \\ \hline
          1 & 1 & 0 \\ \hline
        \end{tabular}
        \endgroup
        \vspace{3pt}
      \end{minipage}
    }\\ \hline

    Ekskluzivno ILI (XOR) &
    \begin{tikzpicture}[circuit logic US, x=1em, y=1em, baseline={(current bounding box.center)}]
      \node[xor gate US, draw, minimum width=3.2em, logic gate inputs=nn] (g) {};
      \draw (g.input 1) -- ++(-0.9,0) node[left] {$A$};
      \draw (g.input 2) -- ++(-0.9,0) node[left] {$B$};
      \draw (g.output) -- ++(0.9,0) node[right] {$Y$};
    \end{tikzpicture}
    & $Y=A \oplus B$ &
    {\scriptsize
      \begin{minipage}{\linewidth}\centering
        \vspace{3pt}
        \begingroup
        \rowcolors{0}{}{} % disable any active row coloring
        \begin{tabular}{|c|c|c|}
          \hline \rowcolor{gray!40} $A$ & $B$ & $Y$ \\ \hline
          0 & 0 & 0 \\ \hline
          0 & 1 & 1 \\ \hline
          1 & 0 & 1 \\ \hline
          1 & 1 & 0 \\ \hline
        \end{tabular}
        \endgroup
        \vspace{3pt}
      \end{minipage}
    }\\ \hline

    Ekvivalencija (XNOR) &
    \begin{tikzpicture}[circuit logic US, x=1em, y=1em, baseline={(current bounding box.center)}]
      \node[xnor gate US, draw, minimum width=3.2em, logic gate inputs=nn] (g) {};
      \draw (g.input 1) -- ++(-0.9,0) node[left] {$A$};
      \draw (g.input 2) -- ++(-0.9,0) node[left] {$B$};
      \draw (g.output) -- ++(0.9,0) node[right] {$Y$};
    \end{tikzpicture}
    & $Y=\lnot(A \oplus B) = \overline{A \oplus B}$ &
    {\scriptsize
      \begin{minipage}{\linewidth}\centering
        \vspace{3pt}
        \begingroup
        \rowcolors{0}{}{} % disable any active row coloring
        \begin{tabular}{|c|c|c|}
          \hline \rowcolor{gray!40} $A$ & $B$ & $Y$ \\ \hline
          0 & 0 & 1 \\ \hline
          0 & 1 & 0 \\ \hline
          1 & 0 & 0 \\ \hline
          1 & 1 & 1 \\ \hline
        \end{tabular}
        \endgroup
        \vspace{3pt}
      \end{minipage}
    }\\ \hline

  \end{tabularx}
\end{table}

\pagebreak

Jedan od bitnijih koraka u procesu sinteze kombinacionih mreža predstavlja proces
minimizacije logičkih funkcija. Iako su razvijene različite metode koje imaju za cilj da
minimizuju broj logičkih kola potrebnih za sintezu neke logičke funkcije, za potrebe računskih
vežbi iz ovog kursa u svrhu minimizacije logičkih funkcija koristićemo metodu baziranu na
Karnoovim kartama. Zbog toga, za potpuno razumevanje primera koji slede, neophodno je
dobro poznavanje ove oblasti koja je odrađena na časovima predavanja.

Pored poznavanja funkcionalnosti osnovnih logičkih kola, kao i poznavanja metoda za
minimizaciju logičkih funkcija, u toku izrade zadataka potrebno je poznavanje određenih
algebarskih pravila koji važe u slučaju Bulove algebre. U tabeli~\ref{tab:pravila-logike} predstavljan su neka od
osnovnih pravila Bulove algebre koja će biti korišćena kroz izradu zadataka.


\begin{table}[!ht]
  \centering
  \caption{Neka od osnovnih pravila Bulove algebre}
  \label{tab:pravila-logike}
  \begin{tabularx}{\linewidth}{| M{2.5cm} | C | C |}
    \hline
    \rowcolor{gray!40}
    \textbf{Broj pravila} & \textbf{I forma} & \textbf{ILI forma} \\
    \hline
    1  & $0\cdot A = 0$                                  & $0 + A = A$ \\ \hline
    2  & $1\cdot A = A$                                  & $1 + A = 1$ \\ \hline
    3  & $A\cdot A = A$                                  & $A + A = A$ \\ \hline
    4  & $A\cdot \bar{A} = 0$                            & $A + \bar{A} = 1$ \\ \hline
    5  & $A\cdot B = B\cdot A$                           & $A + B = B + A$ \\ \hline
    6  & $A\cdot(B\cdot C) = (A\cdot B)\cdot C$          & $A + (B + C) = (A + B) + C$ \\ \hline
    7  & $A + B\cdot C = (A + B)\cdot(A + C)$            & $A\cdot(B + C) = A\cdot B + A\cdot C$ \\ \hline
    8  & $A\cdot(A + B) = A$                             & $A + A\cdot B = A$ \\ \hline
    9  & $\overline{A\cdot B} = \bar{A} + \bar{B}$       & $\overline{A + B} = \bar{A}\cdot \bar{A}$ \\ \hline
    10 & \multicolumn{2}{|c|}{$A = \overline{\overline{A}}$} \\ \hline
  \end{tabularx}
\end{table}

U drugom delu računskih vežbi obrađivaćemo jednu vrstu hazarda karakterističnih za
sekvencijalne mreže. Ta vrsta hazarda naziva statički hazardi i manifestuje se pojavom lažnih
nula i jedinica na izlazu logičkog kola. U okiru praktičnih primera predstavićemo pristup koji
se može iskoristiti za uklanjanje statičkih hazarda. Međutim, treba imati u vidu da se u većini
praktičnih situacija statički hazardi rešavaju korišćenjem metoda simulacije digitalnih kola.

\pagebreak

%========================
\section{Zadaci}

\subsection{Zadatak}
\noindent
Funkciju
\begin{equation}
Y = (A + C + D)\cdot(\bar{A} + C + \bar{D}) \cdot (B + \bar{C})
\label{eq:zad21}
\end{equation}
realizovati u što minimalnijoj formi:

\begin{enumerate}[label=\alph*) , leftmargin=*, itemsep=2pt]
  \item Korišćenjem osnovnih logičkih kola sa proizvoljnim brojem ulaza
  \item Korišćenjem isključivo NI logičkih kola sa proizvoljnim brojem ulaza
  \item Korišćenjem isključivo NI logičkih kola sa dva ulaza
  \item Korišćenjem isključivo NILI logičkih kola sa dva ulaza
\end{enumerate}

\subsubsection*{Rešenje:} 

a) Pošto se u okviru postavke zadatka traži realizacija u što minimalnijoj formi neophodno
je izvršiti minimizaciju logičke funkcije korišćenjem nekih od dostupnih metoda za
minimizaciju. U okviru ovog kursa, koristićemo metodu Karnoovih karti. Rezultat dobijen
primenom ove metode garantuje minimalnost po pitanju broja logičkih kola ali i po pitanju
broja ulaza.


\begin{figure}[!ht]
  \centering
  \captionsetup{skip=-8pt} % space between K-map and caption (adjust)
\begin{karnaugh-map}(label=corner)[4][4][1][$D$][$C$][$B$][$A$]
  \minterms{1,5,6,7,8,12,14,15}
  \maxterms{0,2,3,4,9,10,11,13}
  % no \implicant lines -> no groups drawn
\end{karnaugh-map}
  \caption{Sadržaj Karnoove karte za funkciju~\eqref{eq:zad21}}
  \label{fig:kmap-21}
\end{figure}


\begin{figure}[!ht]
  \centering
  \captionsetup{skip=4pt} % space between K-map and caption (adjust)

  % Left: grouped zeros
  \begin{subfigure}[t]{0.48\linewidth}
    \centering
    % tighten just this subfigure (optional override)
    \captionsetup{skip=-8pt, belowskip=2pt}
    \begin{karnaugh-map}(label=corner)[4][4][1][$CD$][$AB$]
      \minterms{1,5,6,7,8,12,14,15}
      \maxterms{0,2,3,4,9,10,11,13}

      % --- Example zero groups (choose any valid cover you prefer) ---
      \implicant{0}{4}
      \implicantedge{3}{2}{11}{10}
      \implicant{13}{9}
    \end{karnaugh-map}
    \caption{Proizvod zbirova}
    \label{fig:kmap-zeros}
  \end{subfigure}
  \hfill
  % Right: grouped ones
  \begin{subfigure}[t]{0.48\linewidth}
    \centering
    % tighten just this subfigure (optional override)
    \captionsetup{skip=-8pt, belowskip=2pt}
    \begin{karnaugh-map}(label=corner)[4][4][1][$CD$][$AB$]
      \minterms{1,5,6,7,8,12,14,15}
      \maxterms{0,2,3,4,9,10,11,13}

      % --- A minimal cover of ones (example groups) ---
      \implicant{7}{14}
      \implicant{12}{8}
      \implicant{1}{5}
    \end{karnaugh-map}
    \caption{Zbir proizvoda}
    \label{fig:kmap-ones}
  \end{subfigure}

  \caption{Sadržaj Karnoove karte za funkciju~\eqref{eq:zad21}, gde je a) označena oblast za minimizaciju u formi proizvoda zbirova b)
označena oblast za minimizaiciju u formi zbira proizvoda}
  \label{fig:kmap-both}
\end{figure}


Nakon minimizacije dobijaju se minimalne funkcije u formi zbira proizvoda~\eqref{eq:zad21_min_zp} i u
formi proizvoda zbirova~\eqref{eq:zad21_min_pz}. Za realizaciju funkcije~\eqref{eq:zad21_min_zp} potrebna su 3 I i 2 ILI logička
kola. Isti broj kola potreban je i za realizaciju funcije~\eqref{eq:zad21_min_pz}. Dakle, obe minimalne forme daju
isti broj logičkih kola.

\begin{equation}
Y_{MIN\_ZP} = \bar{A} \bar{C} D + A \bar{C} \bar{D} + BC
\label{eq:zad21_min_zp}
\end{equation}

\begin{equation}
Y_{MIN\_PZ} = (A + C + D)(\bar{A} + C + \bar{D})( B + \bar{C})
\label{eq:zad21_min_pz}
\end{equation}


\noindent
Na slici~\ref{fig:zadatak1-zp} je predstavljena realizacija funkcije~\eqref{eq:zad21_min_zp}.

\begin{figure}[!ht]
  \centering
  \includegraphics[width=0.7\linewidth]{Zadatak_1/Zadatak1_ZP}% adjust path if needed
  \caption{Implementacija funkcije~\eqref{eq:zad21} u formi zbira proizvoda.}
  \label{fig:zadatak1-zp}
\end{figure}

\noindent\makebox[\linewidth]{\dotfill}

b) U okviru ove tačke zahtevana je implementacija funkcije~\eqref{eq:zad21} korišćenjem isključivo
NI kola. Sistematski pristup za implementaciju funkcije koja sadrži isključivo NI kola
podrazumeva dva koraka:
\begin{enumerate}
\item[$1)$] Transformaciju minimalne funkcije u formi zbira proizvoda u algebarski zapis koji
sadrži isključivo komplementirane proizvode
\item[$2)$] Šemu koja sadrži isključivo NI kola a koja se dobija na osnovu algebarskog izraza
dobijenog u koraku 1
\end{enumerate}

\vspace{1em}
Korak (1) obuhvata niz transformacija koje za cilj imaju dobijanje algebarskog zapisa u kome
isključivo figurišu komplementirani proizvodi. U slučaju da se zahteva realizacija isključivo
korišćenjem NI kola sa n ulaza, proizvodi moraju sadržati tačno n činioca. U okviru ove tačke
broj ulaza nije definisan tako da ne postoji ograničenje po pitanju broja dostupnih ulaza u NI
logičko kolo. Kao polazni izraz, nad kojim će se vršiti dalje transformacije, koristi se formazbira proizvoda funkcije dobijene primenom Karnoove karte. Za slučaj funkcije~\eqref{eq:zad21} polazni
izraz predstavlja~\eqref{eq:zad21_min_zp} odnosno zbir proizvoda. Na osnovu pravila Bulove algebre, koji kaže da se vrednost funkcije
ne menja dvostrukim komplementiranjem iste, dobijamo:


\begin{equation}
Y = \overline{\overline{Y}} = \overline{\overline{\bar{A} \bar{C} D + A \bar{C} \bar{D} + BC}}
\label{eq:zad21_min_zp_1}
\end{equation}
Daljim transformacija kao krajnji rezultat dobijamo:
\begin{equation}
Y = \overline{\overline{Y}} = \overline{\color{red}{\overline{\bar{A} \bar{C} D}} \cdot \color{blue}{\overline{A \bar{C} \bar{D}}} \cdot \color{ForestGreen}{\overline{BC}}}
\label{eq:zad21_min_zp_2}
\end{equation}

\noindent
Iz funkcije~\eqref{eq:zad21_min_zp_2} možemo uočiti da u njoj figurišu isključivo
komplementirani proizvodi, tačnije četiri takva proizvoda (crveni -- $\color{red}{\overline{\bar{A} \bar{C} D}}$;
plavi -- $\color{blue}{\overline{A \bar{C} \bar{D}}}$; zeleni -- $\color{ForestGreen}{\overline{BC}}$; i na kraju globalni
komplement koji predstavlja celu funkciju).


\vspace{1em}
\noindent
Na slici~\ref{fig:zadatak1-proiz-ni} je predstavljena implementacija funkcije~\eqref{eq:zad21_min_zp_2}.

\begin{figure}[!ht]
  \centering
  \includegraphics[width=0.7\linewidth]{Zadatak_1/Zadatak1_NIProizInput}% adjust path if needed
  \caption{Implementacija funkcije~\eqref{eq:zad21} korišćenjem isključivo NI kola sa proizvoljnim brojem ulaza.}
  \label{fig:zadatak1-proiz-ni}
\end{figure}

\noindent\makebox[\linewidth]{\dotfill}

c) U okviru ove tačke se takođe zahteva realizacija funkcije~\eqref{eq:zad21} korišćenjem \textbf{isključivo
dvoulaznih NI} kola. Koraci (1) i (2) su identični kao u prethodnoj tački što podrazumeva
da i ovde polazimo od algebarskog zapisa funkcije~\eqref{eq:zad21} u formi zbira proizvoda~\eqref{eq:zad21_min_zp}. Nakon niza transformacija~\eqref{eq:zad21_min_2ni_1}-\eqref{eq:zad21_min_2ni_4} dobijamo krajnji izraz~\eqref{eq:zad21_min_2ni_5}.

\begin{equation}
Y = \bar{C} (\bar{A}  D + A \bar{D}) + BC
\label{eq:zad21_min_2ni_1}
\end{equation}

\begin{equation}
Y = \overline{\overline{Y}} =  \overline{\overline{\bar{C} (\bar{A}  D + A \bar{D}) + BC}}
\label{eq:zad21_min_2ni_2}
\end{equation}

\begin{equation}
Y = \overline{\overline{\bar{C} (\bar{A} D + A \bar{D})} \cdot \overline{BC}}
\label{eq:zad21_min_2ni_3}
\end{equation}

\begin{equation}
Y = \overline{
\overline{\bar{C} \overline{\overline{(\bar{A} D + A \bar{D})}}}
\cdot
\overline{BC}
}
\label{eq:zad21_min_2ni_4}
\end{equation}

\begin{equation}
Y = \overline{
\color{orange}{\overline{\bar{C} 
\color{ForestGreen}{\overline{
\color{black}{(}\color{red}{\overline{\bar{A} D}}
\color{ForestGreen}{ \cdot }
\color{blue}{\overline{A \bar{D}}}\color{black}{)}
}}
}}
\color{black}{ \cdot }
\color{Plum}{
\overline{BC}
}}
\label{eq:zad21_min_2ni_5}
\end{equation}

\noindent
Na slici~\ref{fig:zadatak1-2-ni} je predstavljena implementacija funkcije~\eqref{eq:zad21_min_2ni_5}.

\begin{figure}[!ht]
  \centering
  \includegraphics[width=0.7\linewidth]{Zadatak_1/Zadatak1_NI2Ulaza}% adjust path if needed
  \caption{Implementacija funkcije~\eqref{eq:zad21} korišćenjem isključivo dvoulaznih NI kola.}
  \label{fig:zadatak1-2-ni}
\end{figure}

\FloatBarrier   % place all pending floats before carrying on

\noindent\makebox[\linewidth]{\dotfill}

d) U okviru ove tačke zahtevana je implementacija funkcije~\eqref{eq:zad21} korišćenjem isključivo
dvoulaznih NILI kola. Sistematski pristup za implementaciju funkcije koja sadrži isključivo
dvoulazna NILI kola podrazumeva sledeća dva koraka:
\begin{itemize}
\item[$1)$] Transformaciju minimalne funkcije u formi proizvoda zbirova u algebarski zapis koji
sadrži isključivo komplementirane zbirove
\item[$2)$] Šemu koja sadrži isključivo NILI kola a koja se dobija na osnovu algebarskog izraza
dobijenog u koraku 1
\end{itemize}

\vspace{1em}
\noindent
Korak (1) obuhvata niz transformacija koje za cilj imaju dobijanje algebarskog zapisa u kome
isključivo figurišu komplementirani zbirovi sa tačno dva sabirka. Kao polazni izraz, nad kojim
će se vršiti dalje transformacije, koristi se forma proizvoda zbirova funkcije dobijene primenom
Karnoove karte. Za slučaj funkcije~\eqref{eq:zad21} polazni izraz predstavlja~\eqref{eq:zad21_min_pz} ili ponovljeno ispod:

\begin{equation}
Y = (A + C + D)(\bar{A} + C + \bar{D})( B + \bar{C})
\label{eq:zad21_min_pz_again}
\end{equation}

Na osnovu primena pravila 7 Bulove algebre definisanog u Tabeli~\ref{tab:pravila-logike} moguće je
transformisati~\eqref{eq:zad21_min_pz_again} u:

\begin{equation}
Y = (C + (A + D)(\bar{A} + \bar{D}))( B + \bar{C})
\label{eq:zad21_pz_tran_1}
\end{equation}

Duplim komplementiranje izraza~\eqref{eq:zad21_pz_tran_1} dobija se

\begin{equation}
Y = \overline{\overline{Y}} = \overline{\overline{ (C + (A + D)(\bar{A} + \bar{D}))( B + \bar{C}) }}
\label{eq:zad21_pz_tran_2}
\end{equation}

\begin{equation}
Y = 
\overline{
\overline{ (C + (A + D)(\bar{A} + \bar{D}))} 
+ 
\overline{ ( B + \bar{C}) }
}
\label{eq:zad21_pz_tran_3}
\end{equation}

\begin{equation}
Y = 
\overline{
\overline{ (C + 
\overline{ \overline{ (A + D)(\bar{A} + \bar{D})) }}
} 
+ 
\overline{ ( B + \bar{C}) }
}
\label{eq:zad21_pz_tran_4}
\end{equation}

\begin{equation}
Y = 
\overline{
\color{Bittersweet}{ \overline{ (C + 
\color{ForestGreen}{ \overline{ 
\color{red}{ \overline{ (A + D) } }
\color{ForestGreen}{ + }
\color{blue}{ \overline{ (\bar{A} + \bar{D})) } }
}}
} }
\color{black}{ + }
\color{Plum}{ \overline{ ( B + \bar{C}) } }
}
\label{eq:zad21_pz_tran_5}
\end{equation}

\newpage
\noindent
Na slici~\ref{fig:zadatak1-2-nili} je predstavljena implementacija funkcije~\eqref{eq:zad21_pz_tran_5}.

\begin{figure}[!h]
  \centering
  \includegraphics[width=0.7\linewidth]{Zadatak_1/Zadatak1_NILI2Ulaza}% adjust path if needed
  \caption{Implementacija funkcije~\eqref{eq:zad21} korišćenjem isključivo dvoulaznih NILI kola.}
  \label{fig:zadatak1-2-nili}
\end{figure}



\clearpage

\subsection{Zadatak}


\begin{enumerate}[label=\alph*), leftmargin=*, itemsep=2pt]
  \item Odrediti funkcionalnu tabelu za funkciju $Z = f(A,B,C,D)$, gde je $Z$ jednako $1$ kada u zapisu
  četvorobitnog broja $ABCD$ postoje barem dve susedne nule, dok je u suprotnom $Z$ jednako $0$
  (na primer, kada je $ABCD=0000$ tada je $Z=1$, dok kada je $ABCD=0101$ tada je $Z=0$).

  \item Projektovati kombinacionu mrežu koja realizuje funkciju $Z$ ukoliko su dostupna logička
  kola proizvoljnog tipa.
\end{enumerate}

\noindent
\textit{Napomena.} Težiti da broj upotrebljenih logičkih kola bude minimalan.

\subsubsection*{Rešenje:} 

\noindent
a) Sadržaj funkcionalne tabele za funkciju Z je:

\begin{table}[!ht]
\centering
\caption{Funkcionalna tabela za $Z = f(A,B,C,D)$}\label{tab:zad22}
\begin{tabular}{|c|c|c|c|c|}
\hline
\rowcolor{gray!40}
A & B & C & D & Z \\ \hline
0 & 0 & 0 & 0 & 1 \\ \hline
0 & 0 & 0 & 1 & 1 \\ \hline
0 & 0 & 1 & 0 & 1 \\ \hline
0 & 0 & 1 & 1 & 1 \\ \hline
0 & 1 & 0 & 0 & 1 \\ \hline
0 & 1 & 0 & 1 & 0 \\ \hline
0 & 1 & 1 & 0 & 0 \\ \hline
0 & 1 & 1 & 1 & 0 \\ \hline
1 & 0 & 0 & 0 & 1 \\ \hline
1 & 0 & 0 & 1 & 1 \\ \hline
1 & 0 & 1 & 0 & 0 \\ \hline
1 & 0 & 1 & 1 & 0 \\ \hline
1 & 1 & 0 & 0 & 1 \\ \hline
1 & 1 & 0 & 1 & 0 \\ \hline
1 & 1 & 1 & 0 & 0 \\ \hline
1 & 1 & 1 & 1 & 0 \\ \hline
\end{tabular}
\end{table}

\noindent\makebox[\linewidth]{\dotfill}

\noindent
b) Pošto se postavkom zadatka zahteva da broj upotrebljenih kola bude minimalan,
realizaciju ove logičke funkcije potrebno je započeti korišćenjem metode minimizacije bazirane na upotrebi Karnoovih karti. Sadržaj Karnove karte za funkcionalnu tabelu~\ref{tab:zad22} iz
prethodne tačke je prikazan na slici \ref{fig:kmap-adjacent-zeros}.

\begin{figure}[!ht]
  \centering
  \captionsetup{skip=-8pt} % space between K-map and caption (adjust)

  % Z = 1 for minterms: 0,1,2,3,4,8,9,12
  % Z = 0 for minterms: 5,6,7,10,11,13,14,15
  \begin{karnaugh-map}(label=corner)[4][4][1][$D$][$C$][$B$][$A$]
    \minterms{0,1,2,3,4,8,9,12}
    \maxterms{5,6,7,10,11,13,14,15}
    % no groups drawn
  \end{karnaugh-map}

  \caption{Sadržaj Karnoove karte za funkciju $Z$.}
  \label{fig:kmap-adjacent-zeros}
\end{figure}

Funkciju $Z$ ćemo najpre izraziti u obe forme (zbir proizvoda i proizvod zbirova) a zatim ćemo analizom utvrditi koja forma zahteva manji broj logičkih kola.

\begin{figure}[!ht]
  \centering
  \captionsetup{skip=4pt} % main caption spacing

  % ===== Left: group zeros (POS) =====
  \begin{subfigure}[t]{0.48\linewidth}
    \centering
    % tighten just this subfigure (optional override) 
    \captionsetup{skip=-8pt, belowskip=2pt}
    \begin{karnaugh-map}(label=corner)[4][4][1][$CD$][$AB$]
      \minterms{0,1,2,3,4,8,9,12}
      \maxterms{5,6,7,10,11,13,14,15}
      \implicant{5}{15}
      \implicant{7}{14}
      \implicant{15}{10}
    \end{karnaugh-map}
    \caption{Proizvod zbirova}
    \label{fig:kmap-adjzeros}
  \end{subfigure}
  \hfill
  % ===== Right: group ones (SOP) =====
  \begin{subfigure}[t]{0.48\linewidth}
    \centering
    % tighten just this subfigure (optional override) 
    \captionsetup{skip=-8pt, belowskip=2pt}
    \begin{karnaugh-map}(label=corner)[4][4][1][$CD$][$AB$]
      \minterms{0,1,2,3,4,8,9,12}
      \maxterms{5,6,7,10,11,13,14,15}

      \implicant{0}{8}
      \implicant{0}{2}
      \implicantedge{0}{1}{8}{9}
    \end{karnaugh-map}
    \caption{Zbir proizvoda}
    \label{fig:kmap-adjones}
  \end{subfigure}

  \caption{Karnoova mape za funkciju $Z$ gde je a) označena oblast za minimizaciju u formi proizvoda zbirova b)
označena oblast za minimizaiciju u formi zbira proizvoda.}
  \label{fig:kmap-adj-both}
\end{figure}

\noindent
Na osnovu slike slike~\ref{fig:kmap-adj-both} dobijamo sledeće funkcije:

\begin{equation}
Y_{MIN\_ZP} = \bar{C}\bar{D} + \bar{A}\bar{B} + \bar{B}\bar{C}
\label{eq:zad22_min_zp}
\end{equation}

\begin{equation}
Y_{MIN\_PZ} = (\bar{B} + \bar{C})(\bar{B} + \bar{D})(\bar{A} + \bar{C})
\label{eq:zad22_min_pz}
\end{equation}

Na osnovu dobijenih logičkih funkcija možemo zaključiti da je isti broj logičkih kola
potreban za realizaciju obe forme logičke funkcije $Z$. 

\FloatBarrier   % place all pending floats before carrying on

\noindent
Na slici~\ref{fig:zadatak2-impl} je prikazana implementacija
funkcije~\eqref{eq:zad22_min_zp}.
\begin{figure}[!h]
  \centering
  \includegraphics[width=0.65\linewidth]{Zadatak_2/Zadatak2}% adjust path if needed
  \caption{Implementacija funkcije $Z$ korišćenjem forme zbira proizvoda.}
  \label{fig:zadatak2-impl}
\end{figure}

\newpage

\subsection{Zadatak}

Projektovati kombinacionu mrežu koja realizuje izlaz
$
C(C_3 C_2 C_1 C_0) = A(A_1 A_0) \cdot B(B_1 B_0),
$
gde su $A$ i $B$ neoznačeni binarni brojevi. Kombinacionu mrežu je potrebno realizovati
korišćenjem minimalnog broja osnovnih logičkih kola sa proizvoljnim brojem ulaza.

\subsubsection*{Rešenje:} 

Prvi korak u rešavanju ovog zadatka predstavlja kreiranje sadržaja funkcionalne tabele:

\begin{table}[!ht]
\centering
\caption{Funkcionalna tabela za $C = A \cdot B$}\label{tab:zad23}
\begin{tabular}{|c:c||c:c||c:c:c:c|}
\hline
\rowcolor{gray!40}
$A_1$ & $A_0$ & $B_1$ & $B_0$ & $C_3$ & $C_2$ & $C_1$ & $C_0$ \\ \hline\hline
0 & 0 & 0 & 0 & 0 & 0 & 0 & 0 \\ \hline
0 & 0 & 0 & 1 & 0 & 0 & 0 & 0 \\ \hline
0 & 0 & 1 & 0 & 0 & 0 & 0 & 0 \\ \hline
0 & 0 & 1 & 1 & 0 & 0 & 0 & 0 \\ \hline

0 & 1 & 0 & 0 & 0 & 0 & 0 & 0 \\ \hline
0 & 1 & 0 & 1 & 0 & 0 & 0 & 1 \\ \hline
0 & 1 & 1 & 0 & 0 & 0 & 1 & 0 \\ \hline
0 & 1 & 1 & 1 & 0 & 0 & 1 & 1 \\ \hline

1 & 0 & 0 & 0 & 0 & 0 & 0 & 0 \\ \hline
1 & 0 & 0 & 1 & 0 & 0 & 1 & 0 \\ \hline
1 & 0 & 1 & 0 & 0 & 1 & 0 & 0 \\ \hline
1 & 0 & 1 & 1 & 0 & 1 & 1 & 0 \\ \hline

1 & 1 & 0 & 0 & 0 & 0 & 0 & 0 \\ \hline
1 & 1 & 0 & 1 & 0 & 0 & 1 & 1 \\ \hline
1 & 1 & 1 & 0 & 0 & 1 & 1 & 0 \\ \hline
1 & 1 & 1 & 1 & 1 & 0 & 0 & 1 \\ \hline
\end{tabular}
\end{table}

Nakon kreiranja funkcionalne tabele, primenom metode optimizacije bazirane na Karnoovim
kartama, potrebno je realizovati minimalnu funkciju za svaki od izlaznih signala. Sadržaj
Karnoovih karti je prikazan na slikama~\ref{fig:kmap-2x2_1} i~\ref{fig:kmap-2x2_2}.

\begin{figure}[!ht]
  \centering
  \captionsetup{skip=4pt}

  % ========== ROW 1: C3 and C2 ======================================
  \begin{subfigure}[t]{0.48\linewidth}
    \centering
    \captionsetup{skip=-8pt}
    % ================= C3 =====================
    \begin{karnaugh-map}(label=corner)[4][4][1][$B_1B_0$][$A_1A_0$]
      \minterms{15}
      \maxterms{0,1,2,3,4,5,6,7,8,9,10,11,12,13,14}

      % Group of ones (single 1)
      \implicant{15}{15}
    \end{karnaugh-map}
    \caption{Karnoova mapa za $C_3$}
    \label{fig:kmap-C3}
  \end{subfigure}\hfill
  % --------------------------------------------------
  \begin{subfigure}[t]{0.48\linewidth}
    \centering
    \captionsetup{skip=-8pt}
    % ================= C2 =====================
    \begin{karnaugh-map}(label=corner)[4][4][1][$B_1B_0$][$A_1A_0$]
      \minterms{10,11,14}
      \maxterms{0,1,2,3,4,5,6,7,8,9,12,13,15}

      % Groups of ones
      \implicant{11}{10} % horizontal pair
      \implicant{14}{10} % isolated
    \end{karnaugh-map}
    \caption{Karnoova mapa za $C_2$}
    \label{fig:kmap-C2}
  \end{subfigure}


  \caption{Karnoove karte za $C_3$ i$C_2$ proizvoda $C = A \cdot B$, sa grupisanim jedinicama (ZP).}
  \label{fig:kmap-2x2_1}
\end{figure}


\begin{figure}[!ht]
  \centering
  \captionsetup{skip=4pt}

  % ========== ROW 2: C1 and C0 ======================================
  \begin{subfigure}[t]{0.48\linewidth}
    \centering
    \captionsetup{skip=-8pt}
    % ================= C1 =====================
    \begin{karnaugh-map}(label=corner)[4][4][1][$B_1B_0$][$A_1A_0$]
      \minterms{6,7,9,11,13,14}
      \maxterms{0,1,2,3,4,5,8,10,12,15}

      % Groups of ones
      \implicant{7}{6}
      \implicant{6}{14}
      \implicant{13}{9}
      \implicant{9}{11}
    \end{karnaugh-map}
    \caption{Karnoova mapa za $C_1$}
    \label{fig:kmap-C1}
  \end{subfigure}\hfill
  % --------------------------------------------------
  \begin{subfigure}[t]{0.48\linewidth}
    \centering
    \captionsetup{skip=-8pt}
    % ================= C0 =====================
    \begin{karnaugh-map}(label=corner)[4][4][1][$B_1B_0$][$A_1A_0$]
      \minterms{5,7,13,15}
      \maxterms{0,1,2,3,4,6,8,9,10,11,12,14}

      % Groups of ones
      \implicant{5}{15}
    \end{karnaugh-map}
    \caption{Karnoova mapa za $C_0$ }
    \label{fig:kmap-C0}
  \end{subfigure}

  \caption{Karnoove karte za $C_1$ i$C_0$ proizvoda $C = A \cdot B$, sa grupisanim jedinicama (ZP).}
  \label{fig:kmap-2x2_2}
\end{figure}

Na osnovu sadržaja Karnoovih karti, prikazanih na slikama~\ref{fig:kmap-2x2_1} i~\ref{fig:kmap-2x2_2}, moguće je realizovati
svaku od izlaznih funkcija:

\begin{equation}
C_3 = A_1 A_0 B_1 B_0
\end{equation}


\begin{equation}
C_2 
= A_1 B_1 \bar{B_0} + \bar{A_0} A_1 B_1 
= A_1 B_1 (\bar{B_0} + \bar{A_0}) 
= A_1 B_1 \overline{B_0 A_0}
\end{equation}

\begin{equation}
\begin{aligned}
C_1 
&= A_1 \bar{B_1} B_0 + A_1 \bar{A_0} B_0 + \bar{A_1} A_0 B_1 + A_0 B_1 \bar{B_0} \\
&= A_1  B_0  (\bar{A_0} + \bar{B_1}) +  A_0 B_1 ( \bar{A_1} + \bar{B_0} ) \\
&= A_1  B_0  (\overline{A_0 B_1}) +  A_0 B_1 ( \overline{A_1 B_0} ) 
= A_1  B_0 \oplus A_0 B_1
\end{aligned}
\end{equation}

\begin{equation}
C_0 = A_0 B_0
\end{equation}

Na slici 11 je predstavljena implementacija funkcije $C = A \cdot B$.

\begin{figure}[!h]
  \centering
  \includegraphics[width=0.6\linewidth]{Zadatak_3/Zadatak3}% adjust path if needed
  \caption{Implementacija funkcije $C = A \cdot B$.}
  \label{fig:zadatak3}
\end{figure}


\end{document}
