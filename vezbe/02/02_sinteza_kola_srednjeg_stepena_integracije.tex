%========================
% A4 Article Template
%========================
\documentclass[a4paper,11pt]{article}

% Load xcolor ONCE with all options you need
\usepackage[dvipsnames,table]{xcolor}

%---- Color box
\usepackage[most]{tcolorbox}

%---- Page layout & typography
\usepackage[a4paper,margin=2.5cm,headheight=15pt]{geometry}
\usepackage{microtype} % nicer text
\usepackage{setspace}  % optional line spacing
%\onehalfspacing        % uncomment for 1.5 spacing

% Add/replace these in your preamble
\usepackage[utf8]{inputenc}
\usepackage[T1]{fontenc}
\usepackage[serbian]{babel} % latin script variant
\usepackage{lmodern}

% If using XeLaTeX/LuaLaTeX, comment the three lines above and use:
% \usepackage{fontspec}
% \setmainfont{Latin Modern Roman}

%---- for code
\usepackage{minted}

%---- Flush floats
\usepackage{placeins} % preamble

%---- Karnaugh map
\usepackage[label=corner]{karnaugh-map}

%---- Math
\usepackage{amsmath,amssymb,mathtools}
\numberwithin{equation}{section} % eq numbers like (1.1)

%---- Colors & links
\definecolor{Primary}{HTML}{1F77B4} % custom palette
\definecolor{Accent}{HTML}{2CA02C}
\usepackage[colorlinks=true,linkcolor=Primary,citecolor=Primary,urlcolor=Primary]{hyperref}
\usepackage{cleveref} % smart refs: \cref{fig:...}

%---- Figures
\usepackage{graphicx}
\graphicspath{{Images/}} % put images in ./Images
\usepackage{subcaption} % subfigures
\usepackage{caption}
\captionsetup{font=small,labelfont=bf}

%---- Tables
\usepackage{booktabs}   % professional tables
\usepackage{tabularx}   % tables with flexible width
\usepackage{arydshln}   % adds : for dashed vertical lines and \hdashline
\usepackage{multirow}   % optional, merged rows
\usepackage{siunitx}    % alignment for numbers/units
\sisetup{detect-all}

% for the table and logic symbols
\usepackage{array,makecell}
\usepackage{tikz}
\usetikzlibrary{circuits.logic.US,positioning}


%---- Lists (optional nicer lists)
\usepackage{enumitem}
\setlist{nosep}

%---- Header/Footer (optional)
\usepackage{fancyhdr}
\pagestyle{fancy}
\fancyhf{}
\lhead{\textit{Sinteza kombinacionih mre\v{z}a pomo\'{c}u kola srednjeg stepena integracije}}
\rhead{\thepage}

%---- Table settings
\renewcommand{\arraystretch}{1.25}
\renewcommand{\tabularxcolumn}[1]{m{#1}} % X behaves like m{...}
\newcolumntype{C}{>{\centering\arraybackslash}X} % centered X
\newcolumntype{M}[1]{>{\centering\arraybackslash}m{#1}}
%\setlength{\arrayrulewidth}{0.5pt} % optional thicker borders

\renewcommand{\contentsname}{Sadržaj}

%========================
% Document
%========================
\begin{document}


%====================================
% Custom title page (Serbian, A4)
%====================================
\begin{titlepage}
  \thispagestyle{empty}
  \begin{center}

    %--- Top (faculty/department)
    {\Large\bfseries
    UNIVERZITET U BEOGRADU -- ELEKTROTEHNIČKI FAKULTET\\
    KATEDRA ZA ELEKTRONIKU\par}

    \vspace{25mm}

    %--- Middle (course + subtitle)
    {\LARGE\bfseries DIGITALNA ELEKTRONIKA 1\par}
    \vspace{2mm}
    {\large Materijali za računske vežbe\par}

    \vspace{18mm}

    %--- Title block (exercise title + number)
    {\Large\bfseries
    SINTEZA KOMBINACIONIH MREŽA POMOĆU KOLA SREDNJEG
STEPENA INTEGRACIJE\par}
    \vspace{3mm}
    {\large Vežbe 2\par}

    \vfill

%--- Bottom (authors table on left, date on right)
\begin{minipage}{0.62\linewidth}
  \raggedright
  \textbf{Autori:}\\[4pt]
  \small
  \begin{tabular}{@{}l l@{}}
    Nikola Petrović & \href{mailto:p.z.nikola@etf.bg.ac.rs}{\texttt{p.z.nikola@etf.bg.ac.rs}} \\
    Haris Turkmanović & \href{mailto:haris@etf.bg.ac.rs}{\texttt{haris@etf.bg.ac.rs}} \\
  \end{tabular}
\end{minipage}
\hfill
\begin{minipage}{0.34\linewidth}
  \raggedleft
  Beograd, \today
\end{minipage}

  \end{center}
\end{titlepage}

%========================

\tableofcontents
\newpage



%========================
\section{Zadaci}

\subsection{Zadatak 1}\label{sec:zad1}

\begin{enumerate}[label=\alph*) , leftmargin=*, itemsep=2pt]
  \item Predstaviti logičku šemu multipleksera 4/1
  \item Pomoću multipleksera 4/1 iz ta\v{c}ke a) i NE, ILI i I logičkih kola realizovati kombinacionu mrežu
koja generiše izlaze Y\textsubscript{0} i Y\textsubscript{1} definisane logičkih funkcijama predstavljenim u tabeli~\ref{tab:zad1}.

\end{enumerate}

\begin{table}[!ht]
  \centering
  \caption{Tabela logičkih funkcija za zadatak~\ref{sec:zad1}}
  \label{tab:zad1}
  \begin{tabularx}{0.4\linewidth}{| C | C | C |}
    \hline
    \rowcolor{gray!40}
    C\textsubscript{1}C\textsubscript{0} & Y\textsubscript{1} & Y\textsubscript{0} \\ \hline
    00  & $\overline{A \oplus B}$ & $A \oplus B$            \\ \hline
    01  & $A \oplus B$            & $\overline{A \oplus B}$ \\ \hline
    10  & $\bar{A} \bar{B}$       & $\bar{A} {B}$           \\ \hline
    11  & ${A   B}$               & $A \bar{B}$             \\ \hline

  \end{tabularx}
\end{table}

\subsubsection*{Rešenje:} 

\noindent
a) Izlaz Y kombinacione mreže mutlipleksera 4/1, korišćenog za realizaciju u tački b) je
definisan sledećom funkcijom:

\begin{equation}\label{eq:zad1_b}
Y = 
\bar{S_1} \bar{S_0} D_0 + 
\bar{S_1} {S_0} D_1 + 
{S_1} \bar{S_0} D_2 + 
{S_1} {S_0} D_3 
\end{equation}
Na osnovu ove funkcije dobijamo realizaciju prikazanu na slici~\ref{fig:zadatak1-b}.

\begin{figure}[!ht]
  \centering
  \includegraphics[width=0.85\linewidth]{Zadatak_1/Zadatak_1b}% adjust path if needed
  \caption{Implementacija multipleksera 4/1.}
  \label{fig:zadatak1-b}
\end{figure}

\noindent\makebox[\linewidth]{\dotfill}

\noindent
b) Izlaze kombinacione mreže realizovaćemo korišćenjem dva multipleksera 4/1. Ukoliko
funkcije koje se dovode na ulaz multipleksera nisu izražene u formi pogodnoj za realizaciju I,
ILI ili NE logičkim kolima, najpre je neophodno izvesti njihovu odgovarajuću formu. Za slučaj $\overline{A \oplus B}$ i $A \oplus B$ dobijamo ekvivalentne izraze:

\begin{equation}\label{eq:zad1_1}
A \oplus B = A \bar{B} + \bar{A} B
\end{equation}

\begin{equation}\label{eq:zad1_2}
\overline{A \oplus B} = \overline{A \bar{B} + \bar{A} B}
\end{equation}

Nakon izvođenja ekvivalentnih izraza možemo nacrtati kombinacionu mrežu čiji izlazi su
definisani funkcijama predstavljenim u tabeli~\ref{tab:zad1}. Ta kombinaciona mreža je predstavljena na
slici~\ref{fig:zadatak1-a}.

\begin{figure}[!ht]
  \centering
  \includegraphics[width=0.85\linewidth]{Zadatak_1/Zadatak_1a}% adjust path if needed
  \caption{Realizacija kombinacione mreže za zadatak~\ref{sec:zad1}.}
  \label{fig:zadatak1-a}
\end{figure}

\noindent\makebox[\linewidth]{\dotfill}

%\FloatBarrier   % place all pending floats before carrying on
\noindent
VHDL kod koji implementira multiplekser dat jednacinom~\eqref{eq:zad1_b}.
\inputminted[fontsize=\small, frame=single]{vhdl}{code/Zadatak_1/a/zadatak.vhd}

%\newpage
\noindent
Testbench koji simulira multiplekser dat jednacinom~\eqref{eq:zad1_b} je dat \href{https://github.com/pznikola/digitalna_elektronika_1/blob/main/vezbe/02/code/Zadatak_1/a/tb_zadatak.vhd}{OVDE}.

\newpage

%\FloatBarrier   % place all pending floats before carrying on
\noindent
VHDL kod koji implementira funkciju datu u tabeli~\eqref{tab:zad1}.
\inputminted[fontsize=\small, frame=single]{vhdl}{code/Zadatak_1/b/zadatak.vhd}

%\newpage
\noindent
Testbench koji simulira funkciju datu u tabeli~\eqref{tab:zad1} je dat \href{https://github.com/pznikola/digitalna_elektronika_1/blob/main/vezbe/02/code/Zadatak_1/a/tb_zadatak.vhd}{OVDE}.

\end{document}
