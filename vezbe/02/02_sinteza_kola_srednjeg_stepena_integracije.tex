%========================
% A4 Article Template
%========================
\documentclass[a4paper,11pt]{article}

% Load xcolor ONCE with all options you need
\usepackage[dvipsnames,table]{xcolor}

%---- Color box
\usepackage[most]{tcolorbox}

%---- Page layout & typography
\usepackage[a4paper,margin=2.5cm,headheight=15pt]{geometry}
\usepackage{microtype} % nicer text
\usepackage{setspace}  % optional line spacing
%\onehalfspacing        % uncomment for 1.5 spacing

% Add/replace these in your preamble
\usepackage[utf8]{inputenc}
\usepackage[T1]{fontenc}
\usepackage[serbian]{babel} % latin script variant
\usepackage{lmodern}

% If using XeLaTeX/LuaLaTeX, comment the three lines above and use:
% \usepackage{fontspec}
% \setmainfont{Latin Modern Roman}

%---- for code
\usepackage{minted}

%---- Flush floats
\usepackage{placeins} % preamble

%---- Karnaugh map
\usepackage[label=corner]{karnaugh-map}

%---- Math
\usepackage{amsmath,amssymb,mathtools}
\numberwithin{equation}{section} % eq numbers like (1.1)

%---- Colors & links
\definecolor{Primary}{HTML}{1F77B4} % custom palette
\definecolor{Accent}{HTML}{2CA02C}
\usepackage[colorlinks=true,linkcolor=Primary,citecolor=Primary,urlcolor=Primary]{hyperref}
\usepackage{cleveref} % smart refs: \cref{fig:...}

%---- Figures
\usepackage{graphicx}
\graphicspath{{Images/}} % put images in ./Images
\usepackage{subcaption} % subfigures
\usepackage{caption}
\captionsetup{font=small,labelfont=bf}

%---- Tables
\usepackage{booktabs}   % professional tables
\usepackage{tabularx}   % tables with flexible width
\usepackage{arydshln}   % adds : for dashed vertical lines and \hdashline
\usepackage{multirow}   % optional, merged rows
\usepackage{siunitx}    % alignment for numbers/units
\sisetup{detect-all}

% for the table and logic symbols
\usepackage{array,makecell}
\usepackage{tikz}
\usetikzlibrary{circuits.logic.US,positioning}


%---- Lists (optional nicer lists)
\usepackage{enumitem}
\setlist{nosep}

%---- Header/Footer (optional)
\usepackage{fancyhdr}
\pagestyle{fancy}
\fancyhf{}
\lhead{\textit{Sinteza kombinacionih mre\v{z}a pomo\'{c}u kola srednjeg stepena integracije}}
\rhead{\thepage}

% Shrug command

\def\shrug{\texttt{\raisebox{0.75em}{\char`\_}\char`\\\char`\_\kern-0.5ex(\kern-0.25ex\raisebox{0.25ex}{\rotatebox{45}{\raisebox{-.75ex}"\kern-1.5ex\rotatebox{-90})}}\kern-0.5ex)\kern-0.5ex\char`\_/\raisebox{0.75em}{\char`\_}}}

%---- Table settings
\renewcommand{\arraystretch}{1.25}
\renewcommand{\tabularxcolumn}[1]{m{#1}} % X behaves like m{...}
\newcolumntype{C}{>{\centering\arraybackslash}X} % centered X
\newcolumntype{M}[1]{>{\centering\arraybackslash}m{#1}}
%\setlength{\arrayrulewidth}{0.5pt} % optional thicker borders

\renewcommand{\contentsname}{Sadržaj}

%========================
% Document
%========================
\begin{document}


%====================================
% Custom title page (Serbian, A4)
%====================================
\begin{titlepage}
  \thispagestyle{empty}
  \begin{center}

    %--- Top (faculty/department)
    {\Large\bfseries
    UNIVERZITET U BEOGRADU -- ELEKTROTEHNIČKI FAKULTET\\
    KATEDRA ZA ELEKTRONIKU\par}

    \vspace{25mm}

    %--- Middle (course + subtitle)
    {\LARGE\bfseries DIGITALNA ELEKTRONIKA 1\par}
    \vspace{2mm}
    {\large Materijali za računske vežbe\par}

    \vspace{18mm}

    %--- Title block (exercise title + number)
    {\Large\bfseries
    SINTEZA KOMBINACIONIH MREŽA POMOĆU KOLA SREDNJEG
STEPENA INTEGRACIJE\par}
    \vspace{3mm}
    {\large Vežbe 2\par}

    \vfill

%--- Bottom (authors table on left, date on right)
\begin{minipage}{0.62\linewidth}
  \raggedright
  \textbf{Autori:}\\[4pt]
  \small
  \begin{tabular}{@{}l l@{}}
    Nikola Petrović & \href{mailto:p.z.nikola@etf.bg.ac.rs}{\texttt{p.z.nikola@etf.bg.ac.rs}} \\
    Haris Turkmanović & \href{mailto:haris@etf.bg.ac.rs}{\texttt{haris@etf.bg.ac.rs}} \\
  \end{tabular}
\end{minipage}
\hfill
\begin{minipage}{0.34\linewidth}
  \raggedleft
  Beograd, \today
\end{minipage}

  \end{center}
\end{titlepage}

%========================

\tableofcontents
\newpage



%========================
\section{Zadaci}

\subsection{Zadatak 1}\label{sec:zad1}

\begin{enumerate}[label=\alph*) , leftmargin=*, itemsep=2pt]
  \item Predstaviti logičku šemu multipleksera 4/1
  \item Pomoću multipleksera 4/1 iz ta\v{c}ke a) i NE, ILI i I logičkih kola realizovati kombinacionu mrežu
koja generiše izlaze Y\textsubscript{0} i Y\textsubscript{1} definisane logičkih funkcijama predstavljenim u tabeli~\ref{tab:zad1}.

\end{enumerate}

\begin{table}[!ht]
  \centering
  \caption{Tabela logičkih funkcija za zadatak~\ref{sec:zad1}}
  \label{tab:zad1}
  \begin{tabularx}{0.4\linewidth}{| C | C | C |}
    \hline
    \rowcolor{gray!40}
    C\textsubscript{1}C\textsubscript{0} & Y\textsubscript{1} & Y\textsubscript{0} \\ \hline
    00  & $\overline{A \oplus B}$ & $A \oplus B$            \\ \hline
    01  & $A \oplus B$            & $\overline{A \oplus B}$ \\ \hline
    10  & $\bar{A} \bar{B}$       & $\bar{A} {B}$           \\ \hline
    11  & ${A   B}$               & $A \bar{B}$             \\ \hline

  \end{tabularx}
\end{table}

\subsubsection*{Rešenje:} 

\noindent
a) Izlaz Y kombinacione mreže mutlipleksera 4/1, korišćenog za realizaciju u tački b) je
definisan sledećom funkcijom:

\begin{equation}\label{eq:zad1_b}
Y = 
\bar{S_1} \bar{S_0} D_0 + 
\bar{S_1} {S_0} D_1 + 
{S_1} \bar{S_0} D_2 + 
{S_1} {S_0} D_3 
\end{equation}
Na osnovu ove funkcije dobijamo realizaciju prikazanu na slici~\ref{fig:zadatak1-b}.

\begin{figure}[!ht]
  \centering
  \includegraphics[width=0.85\linewidth]{Zadatak_1/Zadatak_1b}% adjust path if needed
  \caption{Implementacija multipleksera 4/1.}
  \label{fig:zadatak1-b}
\end{figure}

\noindent\makebox[\linewidth]{\dotfill}

\noindent
b) Izlaze kombinacione mreže realizovaćemo korišćenjem dva multipleksera 4/1. Ukoliko
funkcije koje se dovode na ulaz multipleksera nisu izražene u formi pogodnoj za realizaciju I,
ILI ili NE logičkim kolima, najpre je neophodno izvesti njihovu odgovarajuću formu. Za slučaj $\overline{A \oplus B}$ i $A \oplus B$ dobijamo ekvivalentne izraze:

\begin{equation}\label{eq:zad1_1}
A \oplus B = A \bar{B} + \bar{A} B
\end{equation}

\begin{equation}\label{eq:zad1_2}
\overline{A \oplus B} = \overline{A \bar{B} + \bar{A} B}
\end{equation}

Nakon izvođenja ekvivalentnih izraza možemo nacrtati kombinacionu mrežu čiji izlazi su
definisani funkcijama predstavljenim u tabeli~\ref{tab:zad1}. Ta kombinaciona mreža je predstavljena na
slici~\ref{fig:zadatak1-a}.

\begin{figure}[!ht]
  \centering
  \includegraphics[width=0.85\linewidth]{Zadatak_1/Zadatak_1a}% adjust path if needed
  \caption{Realizacija kombinacione mreže za zadatak~\ref{sec:zad1}.}
  \label{fig:zadatak1-a}
\end{figure}

\noindent\makebox[\linewidth]{\dotfill}

%\FloatBarrier   % place all pending floats before carrying on
\noindent
VHDL kod koji implementira multiplekser dat jednacinom~\eqref{eq:zad1_b}.
\inputminted[fontsize=\small, frame=single]{vhdl}{code/Zadatak_1/a/zadatak.vhd}

%\newpage
\noindent
Testbench koji simulira multiplekser dat jednacinom~\eqref{eq:zad1_b} je dat \href{https://github.com/pznikola/digitalna_elektronika_1/blob/main/vezbe/02/code/Zadatak_1/a/tb_zadatak.vhd}{OVDE}.

\newpage

%\FloatBarrier   % place all pending floats before carrying on
\noindent
VHDL kod koji implementira funkciju datu u tabeli~\eqref{tab:zad1}.
\inputminted[fontsize=\small, frame=single]{vhdl}{code/Zadatak_1/b/zadatak.vhd}

%\newpage
\noindent
Testbench koji simulira funkciju datu u tabeli~\eqref{tab:zad1} je dat \href{https://github.com/pznikola/digitalna_elektronika_1/blob/main/vezbe/02/code/Zadatak_1/a/tb_zadatak.vhd}{OVDE}.


\newpage

%--------------------------------------------------------------------
\subsection{Zadatak 2}\label{sec:zad2}

\begin{enumerate}[label=\alph*) , leftmargin=*, itemsep=2pt]
  \item Pomoću jednog multipleksera 4/1 i potrebnog broja logičkih kola projektovati kolo koje
generiše izlazni signal Y definisan logičkom funkcijom:
\begin{equation}\label{eq:zad2}
Y = C B \bar{A} + C A + \bar{B} A
\end{equation}
  \item Projektovati kolo u minimalnoj formi ako su dostupna isključivo dvoulazna NI kola.
  \item Uporediti realizaciju pod a) i pod b) u pogledu složenosti i kašnjenja ako je vreme
propagacije signala kroz multiplekser $t_{d-mux}=20$ns a kroz logička kola $t_{d-l} = 10$ns.
\item Da li se u realizaciji pod b) javljaju lažne nule, i ukoliko se javljaju jasno naznačiti pri
kojim prelazima. Predstaviti realizaciju, u vidu logičke funkcije, kojom se otklanjanju lažne
nule.

\end{enumerate}

\subsubsection*{Rešenje:} 

\noindent
a) Prilikom realizacije logičke funkcije koja podrazumeva korišćenje isključivo
multipleksera, za signale na selekcionim linijama S\textsubscript{1}S\textsubscript{0} (u slučaju multipleksera 4/1) treba
postaviti one ulaze koji su u logičkoj funkciji zastupljeni sa najviše različitih kombinacija. U slučaju funkcije~\eqref{eq:zad2} možemo primetiti da su AB i AC podjednako zastupljeni i u nastavku ćemo za selekcione ulaze uzeti AB.

Logičku funkciju odgovarajućeg ulaza D\textsubscript{i} multipleksera 4/1 određujemo primenjujući proceduru koja podrazumeva zamenu promenljivih AB u jednačini:

\begin{equation}\label{eq:zad2-a1}
Y(A, B, C) = C B \bar{A} + C A + \bar{B} A
\end{equation}
sa binarnim kombinacijama od 00 do 11:
\begin{equation}\label{eq:zad2-a2}
\begin{aligned}
Y(0,0,C) &= 0 \\
Y(0,1,C) &= C \\
Y(1,0,C) &= 1 \\
Y(1,1,C) &= C
\end{aligned}
\end{equation}
Na osnovu niza jednakosti~\eqref{eq:zad2-a2} dobijamo realizaciju prikazanu na slici~\ref{fig:zadatak2-a}.

\begin{figure}[!ht]
  \centering
  \includegraphics[width=0.25\linewidth]{Zadatak_2/Zadatak_2a}% adjust path if needed
  \caption{Realizacija funkcije~\eqref{eq:zad2-a1} korišćenjem isključivo multipleksera 4/1 pri \v{c}emu se na selekcione signale multipleksera dovode signali A i B.}
  \label{fig:zadatak2-a}
\end{figure}

\begin{tcolorbox}[colback=gray!10, colframe=black, boxrule=0.5pt]
\textbf{Za samostalni rad:} Implementirati logičku funkciju Y ukoliko su ako selekcioni signali uzeti ulazi AC.
\end{tcolorbox}

\noindent\makebox[\linewidth]{\dotfill}

\noindent
b) Kao i do sada, prvi korak na putu do minimalne realizacije podrazumeva popunjavanje odgovarajuće Karnoove karte na osnovu koje se izvodi funkcija u minimalnoj formi. Sadržaj Karnoove karte za funkciju~\eqref{eq:zad2} je prikazan na slici~\ref{fig:kmap-zad2}.

\begin{figure}[!ht]
  \centering
  \captionsetup{skip=-1em} % space between K-map and caption (adjust)

  \begin{karnaugh-map}[4][2][1][$A$][$B$][$C$]
    \minterms{1,5,6,7}
    \maxterms{0,2,3,4}
    % Groups of ones (prime implicants)
    \implicant{1}{5}   % A \bar{B}
    \implicant{7}{6}   % B C
  \end{karnaugh-map}

  \caption{Karnoova karta za funkciju~\eqref{eq:zad2}.}
  \label{fig:kmap-zad2}
\end{figure}

Pošto se zahteva minimalna realizacija korišćenjem isključivo dvoulaznih NI kola, u Karnoovoj karti tražimo oblasti koje sadrže logičke jedinice, tj. glavni cilj je realizacija funkcije~\eqref{eq:zad2} u formi zbira proizvoda. Na osnovu Karnoove karte sa slike~\ref{fig:kmap-zad2}, dobijamo sledeću funkciju:

\begin{equation}\label{eq:zad2-b1}
Y = C B + \bar{B} A
\end{equation}
Primenom pravila Bulove algebre $Y = \bar{\bar{Y}}$, dobijamo:

\begin{equation}\label{eq:zad2-b2}
Y = \overline{\overline{C B + \bar{B} A}} = 
\overline{
\textcolor{red}{\overline{C B} }
\textcolor{black}{\cdot }
\textcolor{blue}{\overline{\bar{B} A} }
}
\end{equation}
Na osnovu~\eqref{eq:zad2-b2} moguće je realizovati logičku funkciju Y kao što je prikazano na slici~\ref{fig:zadatak2-b}.

\begin{figure}[!ht]
  \centering
  \includegraphics[width=0.35\linewidth]{Zadatak_2/Zadatak_2b.png}% adjust path if needed
  \caption{Realizacija logičke funkcije~\eqref{eq:zad2-b1} pomo\'{c}u NI kola.}
  \label{fig:zadatak2-b}
\end{figure}

\noindent\makebox[\linewidth]{\dotfill}

\noindent
c) Kako se u standardnom 14-pinskom integrisanom kolu mogu naći 4 dvoulazna NI kola, potreban broj integrisanih kola za realizaciju u tački a) i b) je isti i znosi 1. Međutim, sa strane kašnjenja postoje izvesne razlike. U slučaju realizacije iz tačke a) kašnjenje iznosi $t_{d-Y} = t_{d-mux} = 20$ns, dok u slučaju realizacije pod b) kašnjenje iznosi $t_{d-Y} = 3t_{d-l} = 30$ns. Takođe, realizacije pod a) ne zahteva dodatnu minimizaciju date funkcije obzirom da se vrednosti ulaza multipleksera određuju na osnovu konkretnih vrednosti ulaza A i B.

\noindent\makebox[\linewidth]{\dotfill}

\noindent
d) Prelaz pri kojem može doći do pojave lažne nule je označen na slici~\ref{fig:kmap-zad2-d}.


\begin{figure}[!ht]
  \centering
  \captionsetup{skip=-1em} % space between K-map and caption (adjust as needed)
    \begin{tikzpicture}
    % Put the Karnaugh map as a node
    \node (km) {
      \begin{karnaugh-map}[4][2][1][$A$][$B$][$C$]
        \minterms{1,5,6,7}
        \maxterms{0,2,3,4}
        % Groups of ones (prime implicants)
        \implicant{1}{5}   % A \bar{B}
        \implicant{7}{6}   % B C
      \end{karnaugh-map}
    };
 
    \draw[->,blue,line width=1.2pt]
      ([xshift=-3.0cm,yshift=1.45cm]km.south east)  % near cell 14
      -- ++(-0.4cm,0);                             % point toward cell 12
    \draw[<-,blue,line width=1.2pt]
      ([xshift=-3.0cm,yshift=1.65cm]km.south east)  % near cell 14
      -- ++(-0.4cm,0);                             % point toward cell 12
      
  \end{tikzpicture}

  \caption{Karnoova karta funkcije~\eqref{eq:zad2} sa naznačenim prelazima koji dovode do pojave lažne nule.}
  \label{fig:kmap-zad2-d}
\end{figure}
\noindent
U cilju sprečavanja pojave lažne nule, uvodimo dodatnu oblast kao \v{s}to je to prikazano na slici~\ref{fig:kmap-zad2-d1} i dobijamo funkciju:
\begin{equation}\label{eq:zad2-d1}
Y = C B + \bar{B} A + C A
\end{equation}

\begin{figure}[!ht]
  \centering
  \captionsetup{skip=-1em} % space between K-map and caption (adjust)

  \begin{karnaugh-map}[4][2][1][$A$][$B$][$C$]
    \minterms{1,5,6,7}
    \maxterms{0,2,3,4}
    % Groups of ones (prime implicants)
    \implicant{1}{5}
    \implicant{7}{6}
    \implicant{5}{7}
  \end{karnaugh-map}
  \caption{Karnoova karta za funkciju~\eqref{eq:zad2}.}
  \label{fig:kmap-zad2-d1}
\end{figure}
\noindent



\newpage
%--------------------------------------------------------------------
\subsection{Zadatak 3}\label{sec:zad3}

\begin{enumerate}[label=\alph*) , leftmargin=*, itemsep=2pt]
  \item Izvesti funkciju svakog izlaza dekodera 3/8. Izlazi dekodera su aktivni u logičkoj jedinici i dekoder ne sadrži \textit{enable} ulaze
  \item Pomoću dekodera 3/8 sa \textit{enable} ulazom EN\textsubscript{1} aktivnim na nivou logičke jedinice i EN\textsubscript{2}
i EN\textsubscript{3} aktivnim na nivou logičke nule, projektovati dekoder 4/16. Izlazi dekodera 3/8 su aktivni
na nivou logičke nule, dok su ostali ulazi (osim EN\textsubscript{2} u EN\textsubscript{3}) aktivni na nivou logičke jedinice.
Izlazi dekodera 4/16 takođe treba da budu aktivni na nivou logičke nule.
\item Pomoću dekodera 3/8 iz prethodne tačke, i dvoulaznih I logičkih kola, realizovati
sledeću funkciju:
\begin{equation}\label{eq:zad3}
Y = \bar{B} A + C \overline{B A} + \bar{C} B A
\end{equation}
\item Pomoću dekodera iz tačke a) realizovati dekoder 6/64
\end{enumerate}

\subsubsection*{Rešenje:} 

\noindent
a) Tabela~\ref{tab:decoder3to8} predstavlja funkcionalnu tabelu kojom se opisuje način rada dekodera 3/8.


\begin{table}[!ht]
  \centering
  \caption{Funkcionalna tabela dekodera 3/8 u opštem slučaju}
  \label{tab:decoder3to8}
  \begin{tabularx}{0.65\linewidth}{| C | M{0.6cm} | M{0.6cm} | M{0.6cm} | M{0.6cm} | M{0.6cm} | M{0.6cm} | M{0.6cm} | M{0.6cm} |}
    \hline
    \rowcolor{gray!40}
    A\textsubscript{2}A\textsubscript{1}A\textsubscript{0} &
    Y\textsubscript{7} & Y\textsubscript{6} & Y\textsubscript{5} & Y\textsubscript{4} &
    Y\textsubscript{3} & Y\textsubscript{2} & Y\textsubscript{1} & Y\textsubscript{0} \\ \hline
    000 & 0 & 0 & 0 & 0 & 0 & 0 & 0 & 1 \\ \hline
    001 & 0 & 0 & 0 & 0 & 0 & 0 & 1 & 0 \\ \hline
    010 & 0 & 0 & 0 & 0 & 0 & 1 & 0 & 0 \\ \hline
    011 & 0 & 0 & 0 & 0 & 1 & 0 & 0 & 0 \\ \hline
    100 & 0 & 0 & 0 & 1 & 0 & 0 & 0 & 0 \\ \hline
    101 & 0 & 0 & 1 & 0 & 0 & 0 & 0 & 0 \\ \hline
    110 & 0 & 1 & 0 & 0 & 0 & 0 & 0 & 0 \\ \hline
    111 & 1 & 0 & 0 & 0 & 0 & 0 & 0 & 0 \\ \hline
  \end{tabularx}
\end{table}

\noindent
Na osnovu funkcionalne tabele~\ref{tab:decoder3to8} možemo izvesti logičku funkciju svakog od izlaza:
\begin{equation}\label{eq:zad3-a1}
\begin{aligned}
Y_0 = \bar{A_2} \bar{A_1} \bar{A_0} \\
Y_1 = \bar{A_2} \bar{A_1} {A_0} \\
Y_2 = \bar{A_2} {A_1} \bar{A_0} \\
Y_3 = \bar{A_2} {A_1} {A_0} \\
Y_4 = {A_2} \bar{A_1} \bar{A_0} \\
Y_5 = {A_2} \bar{A_1} {A_0} \\
Y_6 = {A_2} {A_1} \bar{A_0} \\
Y_7 = {A_2} {A_1} {A_0} 
\end{aligned}
\end{equation}

%\noindent\makebox[\linewidth]{\dotfill}

\noindent
b) U ovoj tački se zahteva realizacija dekodera 4/16 korišćenjem dekodera 3/8. Za razliku od dekodera 3/8 iz prethodne tačke, ovaj dekoder sadrži tri ulazna enable signala dok su mu izlazi aktivni u logičkoj nuli. Funkcije kojima se opisuju izlazni signali ovog dekodera predstavljaju komplemente funkcija datih jednacinom~\eqref{eq:zad3-a1}.

\begin{figure}[!ht]
  \centering
  \includegraphics[width=0.75\linewidth]{Zadatak_3/Zadatak_3a}% adjust path if needed
  \caption{Realizacija dekodera 4/16 korišćenjem dva dekodera 3/8.}
  \label{fig:zadatak3-a}
\end{figure}

\FloatBarrier   % place all pending floats before carrying on

\noindent\makebox[\linewidth]{\dotfill}

\noindent
c) Pošto se u ovoj tački zahteva realizacija funkcije~\eqref{eq:zad3} korišćenjem dekodera, navedenu funkciju je neophodno izraziti u formi koja sadrži neke od komplementiranih proizvoda~\eqref{eq:zad3-a1}. Do ove forme je najlakše doći analizirajući Karnoovu kartu koja odgovara funkciji~\eqref{eq:zad3} i izdvajanjem izlaza aktivnih u logičkoj nuli koji se formiraju potpunim
zbirovima. Sadržaj Karnoove karte, sa naznačenim oblastima koji se realizuju u formi potpunih zbirova je prikazan na slici 2.3.2


\begin{figure}[!ht]
  \centering
  \captionsetup{skip=-1.8em} % space between K-map and caption (adjust)

  \begin{karnaugh-map}[4][2][1][$A$][$B$][$C$]
    \minterms{1,3,4,5,6}
    \maxterms{0,2,7}
    % Groups of zeros
    \implicant{0}{0} 
    \implicant{7}{7}
    \implicant{2}{2}
  \end{karnaugh-map}
  \caption{Karnoova karta za funkciju~\eqref{eq:zad3}.}
  \label{fig:kmap-zad3-c1}
\end{figure}

\noindent
Na osnovu oblasti označenih na slici~\ref{fig:kmap-zad3-c1}, izvodimo sledeću formu funkcije~\eqref{eq:zad3}:

\begin{equation}\label{eq:zad3-c1}
Y = (C + B + A) (C + \bar{B} + A) (\bar{C} + \bar{B} + \bar{A})
\end{equation}
Ukoliko funkciju~\eqref{eq:zad3-c1} izrazimo koristeći funkcije date jedna\v{c}inom~\eqref{eq:zad3-a1} dobijamo:

\begin{equation}\label{eq:zad3-c2}
Y = \bar{Y_0} \bar{Y_2} \bar{Y_7}
\end{equation}

\noindent
Na osnovu~\eqref{eq:zad3-c2} realizujemo kombinacionu mrežu prikazanu na slici~\ref{fig:zadatak3-c}.

\begin{figure}[!ht]
  \centering
  \includegraphics[width=0.5\linewidth]{Zadatak_3/Zadatak_3b}% adjust path if needed
  \caption{Kombinaciona mreža funkcije~\eqref{eq:zad3} realizovana korišćenjem dekodera 3/8.}
  \label{fig:zadatak3-c}
\end{figure}

\noindent
d) Na slici~\ref{fig:zadatak3-d} je prikazana realizacija dekodera 6/64 korišćenjem dekodera 3/8.

\begin{figure}[!t]
  \centering
  \includegraphics[width=0.6\linewidth]{Zadatak_3/Zadatak_3c}% adjust path if needed
  \caption{Kombinaciona mreža funkcije~\eqref{eq:zad3} realizovana korišćenjem dekodera 3/8.}
  \label{fig:zadatak3-d}
\end{figure}

\vspace{1em}
\textcolor{white}{\shrug}

\clearpage

\newpage

%--------------------------------------------------------------------
\subsection{Zadatak 4}\label{sec:zad4}

\begin{enumerate}[label=\alph*) , leftmargin=*, itemsep=2pt]
  \item Projektovati konvertor BCD koda u sedmobitni kod za pobudu svetlosnog
sedmosegmentnog LED indikatora sa zajedničkom katodom. Ako se na ulazu pojave
kombinacije od 0000 do 1001 na indikatoru se prikazuju cifre 0-9. Ako se na ulazu pojave nedozvoljene kombinacije 1010-1111, na idikatoru se prikazuje slovo E (\textit{Error}). Naraspolaganju su samo NI i NILI logička kola. Težiti da broj upotrebljenih logičkih kola bude minimalan.
\item Projektovati kombinacionu mrežu, ako je poznato da se na ulazu konvertora koda ne mogu pojaviti nedozvoljene kombinacije ulaza, tj. ulazni podatak je u sigurno u opsegu 0000-1001, Na raspolaganju su logička kola proizvoljnog tipa.
\item Kako treba modifikovati dobijenu šemu pod tačkom a) da bi se, u slučaju kada se konvertora koristi u više cifarskom indikatoru, obezbedilo gašenje vodećih nula, i slova E, u slučaju da je na ulazu bilo kog indikatora, nalazi neka od nedozvoljenih kombinacija. Na raspolaganju su logička kola proizvoljnog tipa.
\end{enumerate}

\begin{table}[!ht]
  \centering
  \caption{Primer rada višecifrenog BCD konvertora}
  \label{tab:bcd-konvertor}
  \begin{tabular}{|c|c|}
    \hline
    \rowcolor{gray!40}
    Ulaz u višecifreni BCD konvertor &
    Sadržaj na višecifrenom sedmosegmentnom LE displeju \\ \hline
    0231 & 231 \\ \hline
    00A5 & E   \\ \hline
    0504 & 504 \\ \hline
  \end{tabular}
\end{table}


\begin{figure}[!ht]
  \centering
  \includegraphics[width=0.3\linewidth]{Zadatak_4/7seg}% adjust path if needed
  \caption{Sedmosegmentni displej sa označenim segmentima.}
  \label{fig:zadatak4}
\end{figure}

\newpage

\subsubsection*{Rešenje:} 
\noindent
a) Najpre određujemo funkcionalnu tabelu koja definiše vrednosti izlaznih signala za sve kombinacije ulaznih logičkih nivoa. Na izlazu kombinacione mreže imamo sedmobitni kod (abcdefg) koji se koristi za pobudu LED segmenata. Obzirom da je, prema tekstu zadatka, indikator sa zajedničkom katodom, aktivan logički nivo za pobudu nekog segmenta indikatora je nivo logičke jedinice.

\begin{table}[!ht]
  \centering
  \caption{Funkcionalna tabela uz zadatak~\ref{sec:zad4}}
  \label{tab:bcd-7seg-e}
  \begin{tabularx}{0.7\linewidth}{|C|C|C|C||C|C|C|C|C|C|C|}
    \hline
    \rowcolor{gray!40}
    D & C & B & A & a & b & c & d & e & f & g \\ \hline
    0 & 0 & 0 & 0 & 1 & 1 & 1 & 1 & 1 & 1 & 0 \\ \hline % 0
    0 & 0 & 0 & 1 & 0 & 1 & 1 & 0 & 0 & 0 & 0 \\ \hline % 1
    0 & 0 & 1 & 0 & 1 & 1 & 0 & 1 & 1 & 0 & 1 \\ \hline % 2
    0 & 0 & 1 & 1 & 1 & 1 & 1 & 1 & 0 & 0 & 1 \\ \hline % 3
    0 & 1 & 0 & 0 & 0 & 1 & 1 & 0 & 0 & 1 & 1 \\ \hline % 4
    0 & 1 & 0 & 1 & 1 & 0 & 1 & 1 & 0 & 1 & 1 \\ \hline % 5
    0 & 1 & 1 & 0 & 1 & 0 & 1 & 1 & 1 & 1 & 1 \\ \hline % 6
    0 & 1 & 1 & 1 & 1 & 1 & 1 & 0 & 0 & 0 & 0 \\ \hline % 7
    1 & 0 & 0 & 0 & 1 & 1 & 1 & 1 & 1 & 1 & 1 \\ \hline % 8
    1 & 0 & 0 & 1 & 1 & 1 & 1 & 1 & 0 & 1 & 1 \\ \hline % 9
    % Kodovi veći od 9 -> E (1001111)
    1 & 0 & 1 & 0 & 1 & 0 & 0 & 1 & 1 & 1 & 1 \\ \hline % 10 -> E
    1 & 0 & 1 & 1 & 1 & 0 & 0 & 1 & 1 & 1 & 1 \\ \hline % 11 -> E
    1 & 1 & 0 & 0 & 1 & 0 & 0 & 1 & 1 & 1 & 1 \\ \hline % 12 -> E
    1 & 1 & 0 & 1 & 1 & 0 & 0 & 1 & 1 & 1 & 1 \\ \hline % 13 -> E
    1 & 1 & 1 & 0 & 1 & 0 & 0 & 1 & 1 & 1 & 1 \\ \hline % 14 -> E
    1 & 1 & 1 & 1 & 1 & 0 & 0 & 1 & 1 & 1 & 1 \\ \hline % 15 -> E
  \end{tabularx}
\end{table}

Na osnovu kombinacione tabele~\ref{tab:bcd-7seg-e}, popunjavamo odgovarajuće Karnoove karte a zatim određujemo funkciju svakog od izlaza. U nastavku su date Karnoove karte za svaki od izlaza BCD konvertora.


\begin{figure}[!ht]
  \centering

  %%%%%%%%%%%%%%%%%%%%%%%%%%%%%
  % First row: a, b, c
  %%%%%%%%%%%%%%%%%%%%%%%%%%%%%

  \begin{subfigure}{0.32\textwidth}
    \centering
    \captionsetup{skip=-1em} % space between K-map and caption (adjust)
    % Segment a: 1 for minterms 0,2,3,5,6,7,8,9,10,11,12,13,14,15
    \resizebox{\columnwidth}{!}{%
    \begin{karnaugh-map}[4][4][1][$A$][$B$][$C$][$D$]
      \minterms{0,2,3,5,6,7,8,9,10,11,12,13,14,15}
      \maxterms{1,4}
      \implicant{4}{4}
      \implicant{1}{1}
    \end{karnaugh-map}
    }
    \caption{Segment $a$}
    \label{fig:kmap-a}
  \end{subfigure}
  \hfill
  \begin{subfigure}{0.32\textwidth}
    \centering
    \captionsetup{skip=-1em} % space between K-map and caption (adjust)
    % Segment b: 1 for minterms 0,1,2,3,4,7,8,9
    \resizebox{\columnwidth}{!}{%
    \begin{karnaugh-map}[4][4][1][$A$][$B$][$C$][$D$]
      \minterms{0,1,2,3,4,7,8,9}
      \maxterms{5,6,10,11,12,13,14,15}
      \implicant{12}{14}
      \implicant{5}{13}
      \implicant{15}{10}
      \implicant{6}{14}
    \end{karnaugh-map}
    }
    \caption{Segment $b$}
    \label{fig:kmap-b}
  \end{subfigure}
  \hfill
  \begin{subfigure}{0.32\textwidth}
    \centering
    \captionsetup{skip=-1em} % space between K-map and caption (adjust)
    % Segment c: 1 for minterms 0,1,3,4,5,6,7,8,9
    \resizebox{\columnwidth}{!}{%
    \begin{karnaugh-map}[4][4][1][$A$][$B$][$C$][$D$]
      \minterms{0,1,3,4,5,6,7,8,9}
      \maxterms{2,10,11,12,13,14,15}
      \implicant{12}{14}
      \implicant{15}{10}
      \implicantedge{2}{2}{10}{10}
    \end{karnaugh-map}
    }
    \caption{Segment $c$}
    \label{fig:kmap-c}
  \end{subfigure}

  \vspace{1em}

  %%%%%%%%%%%%%%%%%%%%%%%%%%%%%
  % Second row: d, e, f
  %%%%%%%%%%%%%%%%%%%%%%%%%%%%%

  \begin{subfigure}{0.32\textwidth}
    \centering
    \captionsetup{skip=-1em} % space between K-map and caption (adjust)
    % Segment d: 1 for minterms 0,2,3,5,6,8,9,10,11,12,13,14,15
    \resizebox{\columnwidth}{!}{%
    \begin{karnaugh-map}[4][4][1][$A$][$B$][$C$][$D$]
      \minterms{0,2,3,5,6,8,9,10,11,12,13,14,15}
      \maxterms{1,4,7}
      \implicant{1}{1}
      \implicant{4}{4}
      \implicant{7}{7}
    \end{karnaugh-map}
    }
    \caption{Segment $d$}
    \label{fig:kmap-d}
  \end{subfigure}
  \hfill
  \begin{subfigure}{0.32\textwidth}
    \centering
    \captionsetup{skip=-1em} % space between K-map and caption (adjust)
    % Segment e: 1 for minterms 0,2,6,8,10,11,12,13,14,15
    \resizebox{\columnwidth}{!}{%
    \begin{karnaugh-map}[4][4][1][$A$][$B$][$C$][$D$]
      \minterms{0,2,6,8,10,11,12,13,14,15}
      \maxterms{1,3,4,5,7,9}
      \implicant{4}{5}
      \implicant{1}{7}
      \implicantedge{1}{1}{9}{9}
    \end{karnaugh-map}
    }
    \caption{Segment $e$}
    \label{fig:kmap-e}
  \end{subfigure}
  \hfill
  \begin{subfigure}{0.32\textwidth}
    \centering
    \captionsetup{skip=-1em} % space between K-map and caption (adjust)
    % Segment f: 1 for minterms 0,4,5,6,8,9,10,11,12,13,14,15
    \resizebox{\columnwidth}{!}{%
    \begin{karnaugh-map}[4][4][1][$A$][$B$][$C$][$D$]
      \minterms{0,4,5,6,8,9,10,11,12,13,14,15}
      \maxterms{1,2,3,7}
      \implicant{1}{3}
      \implicant{3}{2}
      \implicant{3}{7}
    \end{karnaugh-map}
    }
    \caption{Segment $f$}
    \label{fig:kmap-f}
  \end{subfigure}

  \vspace{1em}

  %%%%%%%%%%%%%%%%%%%%%%%%%%%%%
  % Third row: g
  %%%%%%%%%%%%%%%%%%%%%%%%%%%%%

  \begin{subfigure}{0.32\textwidth}
    \centering
    \captionsetup{skip=-1em} % space between K-map and caption (adjust)
    % Segment g: 1 for minterms 2,3,4,5,6,8,9,10,11,12,13,14,15
    \resizebox{\columnwidth}{!}{%
    \begin{karnaugh-map}[4][4][1][$A$][$B$][$C$][$D$]
      \minterms{2,3,4,5,6,8,9,10,11,12,13,14,15}
      \maxterms{0,1,7}
      \implicant{0}{1}
      \implicant{7}{7}
    \end{karnaugh-map}
    }
    \caption{Segment $g$}
    \label{fig:kmap-g}
  \end{subfigure}

  \caption{Sadržaj Karnoovih karti za svaki od izlaza BCD konvertora.}
  \label{fig:kmap-7seg}
\end{figure}

Na osnovu sadržaja Karnoovih karti moguće je odrediti funkcije koje podrazumevaju
minimalan broj logičkih kola sa minimalnim brojem ulaza. Svaki od Izlaza logičkog kola je pogodan za implementaciju korišćenjem NILI logičkih kola i zbog toga se u Karnoovim kartama posmatraju logičke nule.

\begin{equation}\label{eq:zad4-a}
\begin{aligned}
a &= (D + \bar{C} + B + A)(D + C + B + \bar{A}) \\[4pt]
b &= (\bar{D} + \bar{B})(\bar{D} + \bar{C})(\bar{C} + \bar{B} + A)(\bar{C} + B + \bar{A}) \\[4pt]
c &= (\bar{D} + \bar{B})(\bar{D} + \bar{C})(C + \bar{B} + A) \\[4pt]
d &= (D + \bar{C} + B + A)(D + C + B + \bar{A})(D + \bar{C} + \bar{B} + \bar{A}) \\[4pt]
e &= (D + \bar{A})(C + B + \bar{A})(D + \bar{C} + B) \\[4pt]
f &= (D + C + \bar{A})(D + C + \bar{B})(D + \bar{B} + \bar{A}) \\[4pt]
g &= (D + C + B)(D + \bar{C} + \bar{B} + \bar{A})
\end{aligned}
\end{equation}

\begin{tcolorbox}[colback=gray!10, colframe=black, boxrule=0.5pt]
\textbf{Za samostalni rad:} Izvesti funkciju za svaki od izlaza ako se umesto NILI logičkih kola
koriste NI logička kola. Uporediti dobijene funkcije sa funkcijama datih u jedna\v{c}ini~\eqref{eq:zad4-a} sa stanovišta broja
logičkih kola.
\end{tcolorbox}

Primenom Bulovih transformacija nad funkcijom svakog od signala potencijalno je
moguće dodatno uprostiti izraze. U slučaju funkcije izlaznog signala \textit{a} važe sledeće jednakosti:

\begin{equation}\label{eq:zad4-a1}
a = (D + \bar{C} + B + A)(D + C + B + \bar{A})
\end{equation}
primenom pravila Bulove algebre dobija se
\begin{equation}\label{eq:zad4-a2}
\begin{aligned}
a &= (D + \bar{C} + B + A)(D + C + B + \bar{A}) \\[4pt]
&=  D + B + (\bar{C} + A)(C + \bar{A}) = D + B + (\bar{C} \bar{A} + C A) \\[4pt]
&= D + B + \overline{C \oplus A}
\end{aligned}
\end{equation}
Na osnovu dobijenog izraza~\eqref{eq:zad4-a2} jasno je da se funkcija a može realizovati sa manjim brojem logičkih kola. Međutim, za takvu realizaciju potrebno je koristiti EXNILI logička kola koja nisu dozvoljena u ovoj tački zadatka.

\begin{tcolorbox}[colback=gray!10, colframe=black, boxrule=0.5pt]
\textbf{Za samostalni rad:} Za ostale signale b-g, primenom Bulovih transformacija, naći formu koja
sadrži manji broj logičkih kola
\end{tcolorbox}

Pošto se u ovoj tački zadatka traži realizacija korišćenjem isključivo NI ili NILI
logičkih kola, potrebno je funkcije izlaza iz Tabele napisati u odgovarajućoj formi. Pošto smo
funkcije, u navedenoj tabeli, izrazili koristeći formu proizvoda zbirova, navedene funkcije je
poželjno realizovati korišćenjem NILI logičkih kola. Zbog toga što broj ulaza nije ograničen
tekstom zadatka, koristićemo NILI kola sa proizvoljnim brojem ulaza. Procedurom sličnom
kao u prethodnim zadacima, dolazimo do forme logičke funkcije izlaznog signala \textit{a} koja je
pogodna za realizaciju NILI logičkim kolima:

\begin{equation}\label{eq:zad4-a3}
\begin{aligned}
a = \bar{\bar{a}} &= \overline{ \overline{ (D + \bar{C} + B + A)(D + C + B + \bar{A}) }} \\[4pt]
&= 
\overline{ 
\textcolor{red}{\overline{ (D + \bar{C} + B + A) }}
\textcolor{black}{+}
\textcolor{blue}{\overline{(D + C + B + \bar{A}) }} }
\end{aligned}
\end{equation}
Na slici~\ref{fig:Zadatak4_a} je ilustrovana implementacija izlaznog signala a korišćenjem isključivo NILI logičkih kola.

\begin{figure}[!ht]
  \centering
  \includegraphics[width=0.8\linewidth]{Zadatak_4/Zadatak4_a}% adjust path if needed
  \caption{Implementacija izlaza a korišćenjem NILI logičkih kola.}
  \label{fig:Zadatak4_a}
\end{figure}

\begin{tcolorbox}[colback=gray!10, colframe=black, boxrule=0.5pt]
\textbf{Za samostalni rad:} 
\begin{enumerate}[leftmargin=*, itemsep=2pt]
\item Za ostale signale b-g izvesti algebarski zapis pogodan za implementaciju korišćenjem
isključivo NILI logičkih kola.
\item Za ostale signale b-g izvesti algebarski zapis pogodan za implementaciju korišćenjem
isključivo NI logičkih kola.
\item Uporediti realizacije sa stanovišta broja logičkih kola
\end{enumerate}
\end{tcolorbox}


\noindent
b) Najpre određujemo kombinacionu tabelu koja definiše vrednosti izlaznih signala za sve kombinacije ulaznih logičkih nivoa. Na izlazu imamo sedmobitni binarni broj (abcdefg). Kako je poznato da na ulazu konvertora može biti samo neka od kombinacija 0000-1001, vrednosti izlaza, u slučaju nedozvoljene kombinacije signala na ulazu kola, mogu imati proizvoljnu vrednost (označenu sa bbbbbbb u funkcionalnoj tabeli) te ih je preporučljivo koristiti prilikom
određivanja logičke funkcije minimalne kompleksnosti.

\begin{table}[!ht]
  \centering
  \caption{Funkcionalna tabela uz zadatak~\ref{sec:zad4} b).}
  \label{tab:bcd-7seg-b}
  \begin{tabularx}{0.7\linewidth}{|C|C|C|C||C|C|C|C|C|C|C|}
    \hline
    \rowcolor{gray!40}
    D & C & B & A & a & b & c & d & e & f & g \\ \hline
    0 & 0 & 0 & 0 & 1 & 1 & 1 & 1 & 1 & 1 & 0 \\ \hline % 0
    0 & 0 & 0 & 1 & 0 & 1 & 1 & 0 & 0 & 0 & 0 \\ \hline % 1
    0 & 0 & 1 & 0 & 1 & 1 & 0 & 1 & 1 & 0 & 1 \\ \hline % 2
    0 & 0 & 1 & 1 & 1 & 1 & 1 & 1 & 0 & 0 & 1 \\ \hline % 3
    0 & 1 & 0 & 0 & 0 & 1 & 1 & 0 & 0 & 1 & 1 \\ \hline % 4
    0 & 1 & 0 & 1 & 1 & 0 & 1 & 1 & 0 & 1 & 1 \\ \hline % 5
    0 & 1 & 1 & 0 & 1 & 0 & 1 & 1 & 1 & 1 & 1 \\ \hline % 6
    0 & 1 & 1 & 1 & 1 & 1 & 1 & 0 & 0 & 0 & 0 \\ \hline % 7
    1 & 0 & 0 & 0 & 1 & 1 & 1 & 1 & 1 & 1 & 1 \\ \hline % 8
    1 & 0 & 0 & 1 & 1 & 1 & 1 & 1 & 0 & 1 & 1 \\ \hline % 9
    % Kodovi veći od 9 -> don't care (b)
    1 & 0 & 1 & 0 & b & b & b & b & b & b & b \\ \hline % 10
    1 & 0 & 1 & 1 & b & b & b & b & b & b & b \\ \hline % 11
    1 & 1 & 0 & 0 & b & b & b & b & b & b & b \\ \hline % 12
    1 & 1 & 0 & 1 & b & b & b & b & b & b & b \\ \hline % 13
    1 & 1 & 1 & 0 & b & b & b & b & b & b & b \\ \hline % 14
    1 & 1 & 1 & 1 & b & b & b & b & b & b & b \\ \hline % 15
  \end{tabularx}
\end{table}

\noindent
Na slici~\ref{fig:kmap-7seg-b} prikazane su Karnoove karte gde su grupisane jedinice. Dobijene funkcije predstavljene u formi zbira proizvoda su date u oviru jedna\v{c}ine~\eqref{eq:zad4-b}.

\begin{figure}[!ht]
  \centering

  %%%%%%%%%%%%%%%%%%%%%%%%%%%%%
  % First row: a, b, c
  %%%%%%%%%%%%%%%%%%%%%%%%%%%%%

  \begin{subfigure}{0.32\textwidth}
    \centering
    \captionsetup{skip=-1em} % space between K-map and caption (adjust)
    \resizebox{\columnwidth}{!}{%
    \begin{karnaugh-map}[4][4][1][$A$][$B$][$C$][$D$]
      \minterms{0,2,3,5,6,7,8,9}
      \maxterms{1,4}
      \terms{10,11,12,13,14,15}{$b$}
            
      \implicant{5}{15}
      \implicant{3}{10}
      \implicant{12}{10}
      \implicantcorner
    \end{karnaugh-map}
    }
    \caption{Segment $a$}
    \label{fig:kmap-a-b}
  \end{subfigure}
  \hfill
  \begin{subfigure}{0.32\textwidth}
    \centering
    \captionsetup{skip=-1em} % space between K-map and caption (adjust)
    \resizebox{\columnwidth}{!}{%
    \begin{karnaugh-map}[4][4][1][$A$][$B$][$C$][$D$]
      \minterms{0,1,2,3,4,7,8,9}
      \maxterms{5,6}
      \terms{10,11,12,13,14,15}{$b$}
            
      \implicantedge{0}{2}{8}{10}
      \implicant{0}{8}
      \implicant{3}{11}
    \end{karnaugh-map}
    }
    \caption{Segment $b$}
    \label{fig:kmap-b-b}
  \end{subfigure}
  \hfill
  \begin{subfigure}{0.32\textwidth}
    \centering
    \captionsetup{skip=-1em} % space between K-map and caption (adjust)
    \resizebox{\columnwidth}{!}{%
    \begin{karnaugh-map}[4][4][1][$A$][$B$][$C$][$D$]
      \minterms{0,1,3,4,5,6,7,8,9}
      \maxterms{2}
      \terms{10,11,12,13,14,15}{$b$}
      
      \implicant{4}{14}
      \implicant{0}{9}
      \implicant{1}{11}
    \end{karnaugh-map}
    }
    \caption{Segment $c$}
    \label{fig:kmap-c-b}
  \end{subfigure}

  \vspace{1em}

  %%%%%%%%%%%%%%%%%%%%%%%%%%%%%
  % Second row: d, e, f
  %%%%%%%%%%%%%%%%%%%%%%%%%%%%%

  \begin{subfigure}{0.32\textwidth}
    \centering
    \captionsetup{skip=-1em} % space between K-map and caption (adjust)
    \resizebox{\columnwidth}{!}{%
    \begin{karnaugh-map}[4][4][1][$A$][$B$][$C$][$D$]
      \minterms{0,2,3,5,6,8,9}
      \maxterms{1,4,7}
      \terms{10,11,12,13,14,15}{$b$}
      
      \implicant{12}{10}
      \implicant{2}{10}
      \implicantedge{3}{2}{11}{10}
      \implicantcorner
      \implicant{5}{13}
    \end{karnaugh-map}
    }
    \caption{Segment $d$}
    \label{fig:kmap-d-b}
  \end{subfigure}
  \hfill
  \begin{subfigure}{0.32\textwidth}
    \centering
    \captionsetup{skip=-1em} % space between K-map and caption (adjust)
    \resizebox{\columnwidth}{!}{%
    \begin{karnaugh-map}[4][4][1][$A$][$B$][$C$][$D$]
      \minterms{0,2,6,8}
      \maxterms{1,3,4,5,7,9}
      \terms{10,11,12,13,14,15}{$b$}
      
      \implicant{2}{10}
      \implicantcorner
    \end{karnaugh-map}
    }
    \caption{Segment $e$}
    \label{fig:kmap-e-b}
  \end{subfigure}
  \hfill
  \begin{subfigure}{0.32\textwidth}
    \centering
    \captionsetup{skip=-1em} % space between K-map and caption (adjust)
    \resizebox{\columnwidth}{!}{%
    \begin{karnaugh-map}[4][4][1][$A$][$B$][$C$][$D$]
      \minterms{0,4,5,6,8,9}
      \maxterms{1,2,3,7}
      \terms{10,11,12,13,14,15}{$b$}
      
      \implicant{12}{10}
      \implicantedge{4}{12}{6}{14}
      \implicant{4}{13}
      \implicant{0}{8}
    \end{karnaugh-map}
    }
    \caption{Segment $f$}
    \label{fig:kmap-f-b}
  \end{subfigure}

  \vspace{1em}

  %%%%%%%%%%%%%%%%%%%%%%%%%%%%%
  % Third row: g
  %%%%%%%%%%%%%%%%%%%%%%%%%%%%%

  \begin{subfigure}{0.32\textwidth}
    \centering
    \captionsetup{skip=-1em} % space between K-map and caption (adjust)
    \resizebox{\columnwidth}{!}{%
    \begin{karnaugh-map}[4][4][1][$A$][$B$][$C$][$D$]
      \minterms{2,3,4,5,6,8,9}
      \maxterms{0,1,7}
      \terms{10,11,12,13,14,15}{$b$}
      
      \implicant{12}{10}
      \implicantedge{3}{2}{11}{10}
      \implicantedge{4}{12}{6}{14}
      \implicant{4}{13}
    \end{karnaugh-map}
    }
    \caption{Segment $g$}
    \label{fig:kmap-g-b}
  \end{subfigure}

  \caption{Sadržaj Karnoovih karti za svaki od izlaza BCD konvertora za zadatak~\ref{sec:zad4} b).}
  \label{fig:kmap-7seg-b}
\end{figure}

\begin{equation}\label{eq:zad4-b}
\begin{aligned}
a &= D + B + C A + \bar{C}\,\bar{A} \\[4pt]
b &= \bar{C} + A B + \bar{A}\,\bar{B} \\[4pt]
c &= C + \bar{B} + A \\[4pt]
d &= D + B\bar{A} + B\bar{C} + \bar{C}\,\bar{A} + C\bar{B}A \\[4pt]
e &= B\bar{A} + \bar{C}\,\bar{A} \\[4pt]
f &= D + C\bar{A} + C\bar{B} + \bar{A}\,\bar{B} \\[4pt]
g &= D + B\bar{C} + C\bar{A} + C\bar{B}
\end{aligned}
\end{equation}

Posmatraju\'{c}i Karnoovu kartu za izlaz \textit{a} koja je data na slici~\ref{fig:kmap-a-b}, izlaz \textit{a} mo\v{z}emo dodatno pojednostaviti:

\begin{equation}\label{eq:zad4-b-a}
\begin{aligned}
a &= D + B + C A + \bar{C}\,\bar{A} \\[4pt]
&= D + B + \overline{C \oplus A}
\end{aligned}
\end{equation}

Implementacija izlaza \textit{a} datog jedna\v{c}inom~\eqref{eq:zad4-b-a} je prikazana na slici~\ref{fig:Zadatak_4b}.

\begin{figure}[!ht]
  \centering
  \includegraphics[width=0.58\linewidth]{Zadatak_4/Zadatak_4b}% adjust path if needed
  \caption{Implementacija funkcije izlaza \textit{a}.}
  \label{fig:Zadatak_4b}
\end{figure}

\begin{tcolorbox}[colback=gray!10, colframe=black, boxrule=0.5pt]
\textbf{Za samostalni rad:} 
Za ostale izlazne signale b-g, implementirati minimalne funkcije. Uporediti dobijene implementacije sa implementacijama u
tački a).
\end{tcolorbox}

\noindent\makebox[\linewidth]{\dotfill}

\noindent
c) Celokupnu funkcionalnost definisanu u okviru ove tačke zadatka moguće je
dekomponovati na sledeće manje zahteve:
\begin{enumerate}
\item Kontrola uključenosti segmenata
\item Detekcija nedozvoljenog ulaza
\item Detekcija vodećih nula
\end{enumerate}
\vspace{1em}

Zahtev (1) podrazumeva uvođenje dodatnog signala u okviru kombinacione mreže
BCD konvertora koji bi omogućio isključivanje svih segmenata LE displeja. Ovaj signal bi
predstavljao ulazni signal u BCD konvertor dok bi logika za kontrolu ovog signala bila
izmeštena izvan samog BCD konvertora. Funkcionalni zahtev (2) podrazumeva uvođenje još
jednog dodatnog izlaznog signala koji ima za cilj prosleđivanje informacije o nevalidnosti
ulaza ostalim BCD konvertorima koji čine višecifreni BCD konvertor. Jedan od
najkopleksnijih funkcionalnih zahteva podrazumeva implementaciju logike za detekciju
vodećih nula koja ima za cilj isključivanje segmenata onih LE displeja koji su povezani na
vodeće nule. Za potpunu implementaciju ove logike neophodno je da do svakog BCD
konvertora, u nizu konvertora koji čine višecifreni BCD konvertor, propagira informacija o
tome da li su prethodni BCD konvertori viših cifara imali niz vodećin nula. Na taj način bi se
omogućilo isključivanje segmenata BCD konvertora koji je poslednji u nizu ukoliko bi se i na
njegovom ulazu pojavio binarni zapis cifre 0.

Na osnovu analize tri glavna funkcionalna zahteva, moguće je identifikovati sledeće
signale u okviru svakog od BCD konvertora:


\begin{table}[!ht]
  \centering
  \caption{Opis upravljačkih signala BCD konvertora}
  \label{tab:signali-bcd}
  \begin{tabularx}{\linewidth}{|c|c|c|X|}
    \hline
    \rowcolor{gray!40}
    Naziv signala & Ulazni/Izlazni & Aktivni logički nivo & Opis \\ \hline

    \texttt{ERROR} & Izlazni & 1 &
    Označava da se na ulazu BCD konvertora pojavila nedozvoljena kombinacija bita. \\ \hline

    \texttt{OFF} & Ulazni & 1 &
    Kontroliše uključenost segmenata. Kada je postavljen na „0“ svi segmenti su isključeni,
    dok je u suprotnom stanje segmenata određeno drugim signalima. \\ \hline

    \texttt{LZ\textsubscript{IN}} & Ulazni & 1 &
    \textit{\textbf{L}eading \textbf{Z}eros} ulazni signal BCD konvertora niže cifre prihvata \textit{\textbf{L}eading \textbf{Z}eros}
    sa BCD konvertora više cifre. Ukoliko su vodeće nule isključene signal je postavljen na jedinicu,
    dok je u suprotnom postavljen na nulu. \\ \hline

    \texttt{LZ\textsubscript{OUT}} & Izlazni & 1 &
    \textit{\textbf{L}eading \textbf{Z}eros} izlazni signal BCD konvertora više cifre se prosleđuje na BCD konvertor niže cifre.
    Ukoliko su vodeće nule isključene signal je postavljen na jedinicu od strane BCD konvertora više cifre,
    dok je u suprotnom postavljen na nulu. \\ \hline

  \end{tabularx}
\end{table}

\texttt{ERROR} signal predstavlja izlazni signal iz jednog BCD konvertora koji se generiše kada
se na ulazu pojavi binarni sadrđaj u ospegu od 1010 -1111. Na osnovu analize sprovedene u
tačkama a) i b) i uvidom u Karnoove karte svakog od izlaznih signala možemo izvesti
minimalnu logičku funkciju \textit{Error} signala:

\begin{equation}
ERROR = D \cdot (C + B)
\end{equation}

Na putu do realizacije kompletne kontrole uključenosti svih segmenata LE displeja,
uvešćemo jedan pomoćni signal \texttt{off\_int} koji je aktivan u logičkoj jedinici. Logika za generisanje ovog internog signala BCD konvertora je prikazana na slici~\ref{fig:Zadatak_4c_OffInternal}.


\begin{figure}[!ht]
  \centering
  \includegraphics[width=0.25\linewidth]{Zadatak_4/Zadatak_4c_OffInternal}% adjust path if needed
  \caption{Logika za kontrolu \texttt{off\_int} signala.}
  \label{fig:Zadatak_4c_OffInternal}
\end{figure}


Do postavljanja ovog signala na nivo logičke jedinice dolazi ukoliko je eksterni signal
\texttt{OFF} postavljen na nivo logičke jedinice ili ukoliko je \texttt{LZ\textsubscript{OUT}} signal postavljen na nivo logičke
jedinice. Drugim rečima, segment koji je kontrolisan \texttt{off\_int} signalom će biti isključen ako se
izvan BCD konvertora aktivira \texttt{OFF} signal ili ukoliko je BCD konvertor u nizu vodeći nula.

Logika za generisanje \texttt{LZ\textsubscript{OUT}} signala se nalazi unutar BCD konvertora i definisana je
izrazom~\eqref{eq:lzout} do je implementacija predstavljena na slici 2.4.7.

\begin{equation}\label{eq:lzout}
LZ_{OUT} = LZ_{IN} \bar{D} \bar{C} \bar{B} \bar{A}
\end{equation}
Dakle, ukoliko su u okviru BCD konvertora viših cifara detektovane vodeće nule (\texttt{LZ\textsubscript{IN}} = 1)
i ako se na ulazu trenutnog BCD konvertora nalazi binarna kombinacija koja odgovara cifri
nula, \texttt{LZ\textsubscript{OUT}} se postavlja na nivo logičke jedinice.

\begin{figure}[!ht]
  \centering
  \includegraphics[width=0.25\linewidth]{Zadatak_4/Zadatak_4c_Lzout}% adjust path if needed
  \caption{Logika za kontrolu \texttt{LZ\textsubscript{OUT}} signala.}
  \label{fig:Zadatak_4c_Lzout}
\end{figure}

Logika ilustrovana na slikama~\ref{fig:Zadatak_4c_OffInternal} i~\ref{fig:Zadatak_4c_Lzout} predstavlja glavne komponente kontrole
uključenosti svakog od segmenata. 

Na slici~\ref{fig:Zadatak_4c_ASegmentControl} je prikazana logika kojom se kontroliše
uključenost segmenta \textit{a}. Kao što se sa slike~\ref{fig:Zadatak_4c_ASegmentControl} može i uočiti, signal \textit{a'} predstavlja dodatno
kontrolisan signal \textit{a} čija je funkcija već predstavljena u prethodnim tačkama.

\begin{figure}[!ht]
  \centering
  \includegraphics[width=0.4\linewidth]{Zadatak_4/Zadatak_4c_ASegmentControl}% adjust path if needed
  \caption{Logika za kontrolu uključenosti segmenta \textit{a}.}
  \label{fig:Zadatak_4c_ASegmentControl}
\end{figure}

Na slici~\ref{fig:Zadatak_4c_CompleteBCDLogic} je predstavljena logička šema BCD konvertora koja implementira set
funkcionalnosti potreban za realizaciju višecifrenog BCD konvertora.


\begin{figure}[!ht]
  \centering
  \includegraphics[width=0.7\linewidth]{Zadatak_4/Zadatak_4c_CompleteBCDLogic}% adjust path if needed
  \caption{Logika jednog BCD konvertora sa dodatnim kontrolnim signalima iz tabele.}
  \label{fig:Zadatak_4c_CompleteBCDLogic}
\end{figure}


\begin{tcolorbox}[colback=gray!10, colframe=black, boxrule=0.5pt]
\textbf{Za samostalni rad:} 
Za ostale izlazne signale \textit{b'-g'} nacrtati odgovarajuće logičke šeme.
\end{tcolorbox}

Logika predstavljena na slici~\ref{fig:Zadatak_4c_CompleteBCDLogic} je enkapsulirana u funkcionalni blok BCD
konvertora koji će biti iskorišćen za sintezu višecifrenog BCD konvertora. Ovaj funcionalni
blok je predstavljen na slici~\ref{fig:Zadatak_4c_BCDComponent}.

\begin{figure}[!ht]
  \centering
  \includegraphics[width=0.28\linewidth]{Zadatak_4/Zadatak_4c_BCDComponent}% adjust path if needed
  \caption{Funkcionalni blok BCD konvertora.}
  \label{fig:Zadatak_4c_BCDComponent}
\end{figure}

Funkcionalni blok predstavljen na slici~\ref{fig:Zadatak_4c_BCDComponent} je iskorišćen za realizaciju višecifrenog BCD konvertora kao što je prikazano na slici~\ref{fig:Zadatak_4c_4Digits}.

\begin{figure}[!ht]
  \centering
  \includegraphics[width=\linewidth]{Zadatak_4/Zadatak_4c_4Digits}% adjust path if needed
  \caption{Višebitni BCD konvertor.}
  \label{fig:Zadatak_4c_4Digits}
\end{figure}

Kao što se sa slike~\ref{fig:Zadatak_4c_4Digits} može videti, \texttt{ERROR} signali svih blokova su dovedeni na
jedno ILI kolo (ozna\v{c}eno na slici~\ref{fig:Zadatak_4c_4Digits} sa 3) što funkcionalno označava da, ukoliko se pojavi nedozvoljena kombinacija
na bilo kom ulazu, generisaće se globalni \texttt{OFF} signal. Ovaj signal kontroliše isključenje svih segmenata u okviru svakog od LE displeja osim displeja najniže cifre višecifrenog BCD konvertora. Najniža cifra je uvek uključena bez obzira šta je na ulazu BCD konvertora viših cifara. Međutim, u slučaju da je \texttt{OFF} signal postavlje na nivo logičke jedinice (što označava da je bilo nedozvoljene kombinacije na nekom od ulaza BCD konvertora koji odgovaraju višim
ciframa), koristeći ILI logička kola (ozna\v{c}ena na slici~\ref{fig:Zadatak_4c_4Digits} sa 1 i 2) doći će do propagiranja te informacije i postavljanja nedozvoljene kombinacije na ulaz BCD konvertora najniže cifre što za posledicu ima ispis slova \texttt{E} na LE displej najniže cifre.

\begin{tcolorbox}[colback=orange!10, colframe=black, boxrule=0.5pt]
\textbf{Napomena:} 
Zbog pojave povratne sprege (na primer signal \texttt{ERROR} je, sa izlaza BCD
konvertora najniže cifre, posredstvom logičkih kola 3, 4, 2 ili 3,4,1 vraćen na ulaz istog BCD konvertora) može doći do pojave ''pamćenja'' vrednosti u povratnoj sprezi koja za posledicu ima nemogućnost promene ispisa na LE displeju najniže cifre. O ovoj pojavi će biti više reči na kursu Digitalna Elektronika 2 . Da bi se kolo izvelo iz navedenog stanja, uvodimo kontrolni signal $\overline{\text{\texttt{INIT}}}$ koji ima za cilj da resetuje vrednost zapamćenu u okviru povratne sprege kombinacione mreže.
\end{tcolorbox}

\section{Zadaci za samostalni rad}

\subsection{Zadatak 1}
\begin{enumerate}[label=\alph*) , leftmargin=*, itemsep=2pt]
  \item Pomoću dekodera $n/2^n$ i $m/2^m$ projektovati dekoder $(n+m)/2^(n+m)$ .Na raspolaganju su dvoulazna logička I kola. Dekodere $n/2^n$ i $m/2^m$ crtati kao blokove.
\item Pomoću dvoulaznih I kola i invertora realizovati dekoder 4/16 kod koga su i ulazni i
izlazni signali aktivni na logičkoj jedinici. Težiti da broj upotrebljenih logičkih kola bude
minimalan.
\item Projektovati kolo dekodera koje ima 13 ulaza tako da realizacija bude izvedena u maksimalno 4 nivoa. Na raspolaganju su doulazna I kola i invertori. Težiti da broj upotrebljenih logičkih kola bude minimalan. Vreme propagiranja kroz logičko kolo iznosi $t_{dlk}$ dok se kašnjenje signala invertora može zanemariti.
\end{enumerate}

\subsection{Zadatak 2}
\noindent
Funkciju:
\begin{equation}
Y = \bar{D} C \bar{A} + \bar{B} A + \bar{C} A
\end{equation}
realizovati isključivo korišćenjem osnovnih logičkih kola sa proizvoljnim brojem ulaza i multipleksera 4/1 za čije selekcione signale S\textsubscript{1}S\textsubscript{0} važi:

\begin{enumerate}[label=\alph*) , leftmargin=*, itemsep=2pt]
  \item $S_1 S_0 = D C$
  \item $S_1 S_0 = B A$
  \item $S_1 S_0 = D B$
  \item $S_1 S_0 = C A$
\end{enumerate}
\noindent
Koja od realizacija a), b), c) i d) zahteva minimalan broj logičkih kola?
\end{document}
